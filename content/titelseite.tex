\newcommand*{\fslogo}{\raisebox{+1.25ex}{\includegraphics[height=6cm]{graphics/logo-fachschaft}}}

	\begin{center}
		\fslogo \\
		\vspace{3em}
		\rule{\textwidth}{1pt}\par
		\vspace{0.8\baselineskip}
        \Huge\bfseries \LaTeX-Tutorium der \\
        Fachschaft WIAI
		\vspace{0.8\baselineskip}
		\rule{\textwidth}{1pt}\par
		%\vspace{2em}
		{\large \today}
		\vfill
		{\Large\textsc{Christian Kremitzl, Florian Knoch,\\
		Bernhard Lüdtke, Anika Amma}}\\		
		\vfill
	\end{center}
