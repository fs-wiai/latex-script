\section{Grundstruktur}

Wie sieht so ein \LaTeX-Dokument jetzt eigentlich aus?
Im Kern besteht es immer aus zwei Teilen: einer Präambel zu Beginn sowie der unmittelbar folgenden Dokumentumgebung.
Als \emph{Präambel} bezeichnen wir die ersten Befehle eines \LaTeX-Dokumentes, mit denen globale Einstellungen wie die Dokumentklasse, die Zeichenkodierung, die Sprache und das Seitenformat festgelegt werden.
Die \emph{Dokumentumgebung} hingegen enthält den Inhalt, der später in der fertigen \acro{PDF}-Datei erscheinen soll.

\Example{lst:grundaufbau-latex-dokument}{grundstruktur/hallowelt}{grundstruktur/hallowelt_crop}{Beispielhafter Aufbau eines einfachen \LaTeX-Dokuments mit Präambel und Dokumentumgebung}

\subsection{Präambel}
Beginnen wir mit der Präambel.
Eine minimale Präambel sollte die folgenden Festlegungen treffen:

\subsubsection{Dokumentklasse}\label{subsubsec:dokumentklasse}
\mintinline{latex}|\documentclass[<parameter>]{<dokumentklasse>}|
definiert die Dokumentklasse. 
Die verbreitetsten von \LaTeX\ standardmäßig unterstützten Dokumentklassen sind \mintinline{latex}{article} für kurze und \mintinline{latex}{report} für längere Dokumente, \mintinline{latex}{book} für Bücher, \mintinline{latex}{beamer}\footnote{Präsentationen in \LaTeX\ sind ein Thema für sich, auf das wir nicht näher eingehen. Bei Interesse sei diese Einführung von Overleaf empfohlen: \url{https://www.overleaf.com/learn/latex/Beamer}} für Präsentationen sowie \mintinline{latex}{letter}\footnote{Auch auf Briefe wird in diesem Skript nicht weiter eingegangen. WikiBooks gibt eine Einführung: \url{https://en.wikibooks.org/wiki/LaTeX/Letters}} für Briefe.

Parallel zu den Standard-Dokumentklassen entwickelten sich die \acro{KOMA}-Script-Klassen.\footnote{Einen umfassenden Überblick über die Möglichkeiten von \acro{KOMA}-Script gibt folgendes Handbuch: \url{https://komascript.de/~mkohm/scrguide.pdf}}
Als Alternative zum \mintinline{latex}{article} steht \mintinline{latex}{scrartcl} bereit, \mintinline{latex}{report} wird durch \mintinline{latex}{scrreport} ersetzt, \mintinline{latex}{scrbook} kann anstelle von \mintinline{latex}{book} verwendet werden und \mintinline{latex}{scrlttr2} ist als Ersatz für \mintinline{latex}{letter} entstanden.
Eine vollständige Liste aller möglichen \acro{KOMA}-Script-Dokumentklassen ist online\footnote{Erreichbar unter \url{https://komascript.de/komascriptbestandteile}} verfügbar.
Die \acro{KOMA}-Klassen ändern das Layout des fertigen \acro {PDF}-Dokuments und bringen zusätzliche Funktionalitäten mit.
Während die Standard-Klassen für den Gebrauch in Amerika gedacht sind, folgen \acro{KOMA}-Klassen europäischen Normen, etwa bei Brief"|formaten.

Dem \mintinline{latex}{\documentclass}-Befehl können verschiedene optionale Parameter übergeben werden. 
\mintinline{latex}{\documentclass[10pt,a5paper,landscape]{scrartcl}} etwa konfiguriert einen \acro{KOMA}-Script-Artikel mit Schriftgröße 10\,pt\footnote{Die Standardschriftgröße wäre hier 12\,pt.} auf A5-Papier\footnote{In der Voreinstellung handelte es sich um A4-Papier.} im Querformat.
Auch die Sprache kann mit übergeben werden (vgl. \cref{subsubsec:sprache}).

\subsubsection{Exkurs: Pakete}
\label{subsub:pakete}
\begin{minted}{latex}
\usepackage[<optionen>]{<paketname>}
\end{minted}
Pakete sind \LaTeX-Erweiterungen, die zusätzliche Befehle und Funktionalitäten zur Verfügung stellen.
Es existieren verschiedene Pakete für verschiedene Anwendungsfälle (z.\,B. Formelsatz, Listen, \textellipsis).
Bevor sie verwendet werden können, müssen sie in der Präambel eingebunden werden. 
Hierfür wird der obige Befehl verwendet.
Die wichtigsten \LaTeX-Pakete sind im Comprehensive \TeX\ Archive Network,\footnote{Erreichbar unter \url{https://www.ctan.org/}} kurz \acro{CTAN}, abgelegt.
Hier finden sich auch Dokumentationen zu den einzelnen Funktionalitäten.

\subsubsection{Kodierung}
\begin{minted}{latex}
\usepackage[utf8]{inputenc}
\usepackage[t1]{fontenc}
\end{minted}
Pakete werden zum Beispiel zur Auswahl der Zeichenkodierung benötigt.
Die Zeichenkodierung\footnote{Vgl. \url{https://de.wikipedia.org/wiki/Zeichenkodierung}} entscheidet darüber, welcher Zeichensatz zur Verfügung steht.
Der Standardzeichensatz in \LaTeX\ ist \acro{ASCII}.\footnote{Vgl. \url{https://de.wikipedia.org/wiki/American_Standard_Code_for_Information_Interchange}}
Dieser amerikanische Standard enthält beispielsweise keine Umlaute und ist daher für die meisten Einsätze wenig geeignet.
Als universelle Zeichenkodierung kann \acro{UTF-8}\footnote{Vgl. \url{https://de.wikipedia.org/wiki/UTF-8}} verwendet werden.

In \LaTeX-Dokumenten sind zwei verschiedene Zeichenkodierungen von Belang.
Das Input Encoding (kurz: \mintinline{latex}{inputenc}) bezieht sich auf den Quelltext, den wir eingeben, während das Font Encoding (kurz: \mintinline{latex}{fontenc}) das fertige \acro{PDF}-Dokument beeinflusst.
Konkret beeinflusst \mintinline{latex}{fontenc} die Wahl der Schriftartdatei, die verwendet wird, um den Inhalt des Dokuments darzustellen, denn auch diese muss natürlich Sonderzeichen beinhalten.\footnote{Details zur Arbeitsweise von \mintinline{latex}{fontenc} gibt dieser Beitrag: \url{https://www.texwelt.de/fragen/5537/was-macht-eigentlich-usepackaget1fontenc}}
\mintinline{latex}{T1} ist eine Kodierung, die mit einer begrenzten Anzahl an Zeichen europäische Sprachen abzubilden versucht.

\subsubsection{Sprache}\label{subsubsec:sprache}
\begin{minted}{latex}
\usepackage[ngerman]{babel}
\end{minted}

Das Paket \mintinline{latex}{babel} stellt sprachspezifische Informationen (z.\,B. Silbentrennung, Sonderzeichen, Wechsel zwischen Fonts, Übersetzung von Labels\footnote{Vgl. \cref{sec:referenzen}} wie \enquote{Kapitel}, \enquote{Inhaltsverzeichnis} oder \enquote{Abbildung}) bereit.
Als optionaler Parameter kann die Sprache übergeben werden.
\mintinline{latex}{ngerman} steht hierbei für die Neue Deutsche Rechtschreibung.
Manche Pakete erfordern es, dass die Sprachangabe schon als optionaler Parameter dem \mintinline{latex}{\documentclass}-Befehl\footnote{Vgl. \cref{subsubsec:dokumentklasse}} übergeben wird.
In diesem Fall wird der optionale Parameter beim Einbinden von \mintinline{latex}{babel} einfach weggelassen.

In einem Dokument können auch mehrere Sprachen verwendet werden.
Sie können einfach kommagetrennt als optionale Parameter aufgelistet werden.
Im Dokument kann mit dem Befehl \mintinline{latex}{\selectlanguage{<sprache>}} zwischen den Sprachen umgeschaltet werden.
Fremdsprachiger Text lässt sich alternativ auch über folgenden Befehl einbinden: 
\begin{minted}{latex}
\foreignlanguage{<sprache>}{<text>}
\end{minted}

\subsection{Dokumentumgebung}

Zwischen \mintinline{latex}{\begin{document}} und \mintinline{latex}{\end{document}} gehören die Inhalte, die tatsächlich im Dokument erscheinen sollen.

\subsubsection{Fließtext}
Der einfachste Inhalt der Dokumentumgebung ist Fließtext.
Er kann direkt in den Quelltext geschrieben werden.
Zeilenumbrüche und zu mehreren auftretende Leerzeichen werden dabei ignoriert.
Leerzeilen erzeugen einen neuen Absatz, der standardmäßig mit einem Einzug beginnt.\footnote{Der Befehl \mintinline{latex}{\noindent} verhindert dies.}
Manuelle Umbrüche können mit zwei Backslashes (\enquote{\textbackslash\textbackslash}) erzwungen werden.
Dies sollte jedoch vermieden werden.

\subsubsection{Kommentare}
Einige Zeichen sind für besondere Funktionen reserviert. 
So bewirkt ein Prozent-Zeichen beispielsweise, dass der Rest einer Zeile vom Compiler ignoriert wird.
Der Text erscheint also nicht im fertigen \acro{PDF}-Dokument.
Dies kann nützlich sein, um während der Bearbeitung Notizen festzuhalten, ohne Einfluss auf das fertige Dokument zu nehmen.
Wir sprechen hier von einem Kommentar.

Falls tatsächlich ein Prozentzeichen erscheinen soll, könnt ihr dies durch einen Backslash erreichen: \mintinline{latex}{\%}.
Diese Lösung, die auch als Escaping bezeichnet wird, funktioniert auch für die Zeichen \#, \$, \&, \_, \{ und \}.
Für den Backslash muss der Befehl \mintinline{latex}{\textbackslash} verwendet werden.\footnote{Eine Übersicht weiterer Sonderzeichen ist im zugehörigen \cref{sec:sonderzeichen} zu finden.}

\subsubsection{Abschnitte und Kapitel}
Texte werden durch Überschriften in Abschnitte und Kapitel unterteilt.
Immer nutzbar sind die in \cref{lst:ueberschriften} dargestellten Befehle.
\Example{lst:ueberschriften}{grundstruktur/ueberschriften}{grundstruktur/ueberschriften_crop}{Überschriftenebenen}
Je nach Dokumentklasse sind außerdem \mintinline{latex}{\chapter{Kapitel}} und \mintinline{latex}{\part{Teil}} verfügbar -- beispielsweise in Büchern.
Mit Sternchen fallen bei allen diesen Befehlen die Nummerierung und der Eintrag im Inhaltsverzeichnis\footnote{Vgl. \cref{subsubsec:inhaltsverzeichnis}} weg: 

\begin{minted}{latex}
\section*{Kommt nicht im Inhaltsverzeichnis vor}
\end{minted}

Ein alternativer Titel im Inhaltsverzeichnis kann als optionaler Parameter in eckigen Klammern zwischen dem Befehl und dem eigentlichen Titel angegeben werden. 

\begin{minted}{latex}
\section[Titel im Verzeichnis]{Kapitelüberschrift}
\end{minted}

\subsubsection{Titelei}
Ein einfacher Titel lässt sich im Dokument mit dem Befehl \mintinline{latex}{\maketitle} erzeugen.
Die Werte für die Einträge müssen in der Präambel festgelegt werden. 
Mehrere Autoren werden dabei mit \mintinline{latex}{\and} verbunden.
Fehlt der Befehl \mintinline{latex}{\date}, wird per default das aktuelle Datum gesetzt.
Die genaue Gestaltung hängt ab von der verwendeten Dokumentklasse.

\Example{lst:titelei}{grundstruktur/titelei}{grundstruktur/titelei_crop}{Die Titelei}

\subsubsection{Verzeichnisse}\label{subsubsec:inhaltsverzeichnis}

Aus allen mit den oben genannten Befehlen eingefügten Abschnitten fügt der Befehl \mintinline{latex}{\tableofcontents} ein automatisch nummeriertes Inhaltsverzeichnis zusammen (wie zum Beispiel in \cref{lst:hauptdatei} auf \cpageref{lst:hauptdatei} zu sehen).
Die Aufzählungszeichen, die Tiefe der Aufzählung und viele weitere Gestaltungselemente sind frei anpassbar.\footnote{Ein empfehlenswerter Blogpost hierzu ist unter \url{https://texblog.org/2011/09/09/10-ways-to-customize-tocloflot/} verfügbar.}
Damit \LaTeX\ das Inhaltsverzeichnis vollständig baut, muss das Projekt zweimal kompiliert werden.

Neben dem Inhaltsverzeichnis stehen auch eine \mintinline{latex}{\listoffigures} (Abbildungsverzeichnis) und eine \mintinline{latex}{\listoftables} (Tabellenverzeichnis) zur Verfügung.
Aufgeführt werden jeweils die Bild- oder Tabellenunterschriften.\footnote{Siehe hierzu \cref{sec:grafiken_einbinden} (Grafiken) und \cref{sec:tabellen} (Tabellen).}
