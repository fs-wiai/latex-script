\chapter{References and Footnotes}
\section{Footnotes}\label{sec:footnotes}


Whenever we want to include footnotes into our \LaTeX{} document, we can use 
the command \code{latex}{\textbackslash footnote{<text>}}. At the position where we use 
the command, the correct number will be inserted automatically, and the text 
within the curly braces will appear in the footer. In combination with the 
package \pkg{hyperref}, \acro{URL}s within footnotes become 
clickable.\footnote{if we use the command \code{latex}{\textbackslash url{…}}}
We can see examples for that in the whole document.

The package \pkg{footmisc} provides us with additional options for how to display footnotes. They can be passed as optional parameters to the command \code{latex}{\textbackslash usepackage}:
\begin{itemize}
  \item \code{latex}{\textbackslash usepackage[perpage]\{footmisc\}} ensures that the count of footnotes begins at 1 for each new page.
  \item \code{latex}{\textbackslash usepackage[para]\{footmisc\}} lets the footnotes within the footer appear as continuous text (i.\,e., the footnotes can also appear next to each other).
  \item \code{latex}{\textbackslash usepackage[symbol]\{footmisc\}} causes a numbering with symbols (e.\,g.,  \textdagger, \textdaggerdbl) instead of numbers.
\end{itemize}


\section{References}\label{sec:references}

If we want to make references, like \enquote{\textellipsis , which you can see 
in figure 21, \textellipsis}, \LaTeX{} by default provides us with the command 
\code{latex}{\textbackslash ref\{<label>\}}. No more adapting of the numbering 
for graphics, tables, etc. needed!
The command expects a unique label as argument, that needs to be assigned to 
the referenced element. After that, wherever we call the command, the number of 
our referenced object appears in the text.
\Example{lst:fooboar-learning-latex}{references/ref}{references/ref_crop}{Example
 for a reference}

A smarter package for references is \pkg{cleverref}.\footnote{with only one 
\enquote{r}!}
It provides us with the command \code{latex}{\textbackslash cref\{<label>\}}, 
which can also handle multiple labels separated by commas.
This automatically generates elegant references like \enquote{sections 1 to 3, 
and 5.}\footnote{for the source code \code{latex}{\textbackslash 
cref\{sec:section1,sec:section2,
		sec:section3,sec:section5\}}}
Furthermore, \code{latex}{\textbackslash ref\{<label>\}} automatically inserts 
a suited abbreviation, e.\,g., \enquote{fig.} for figures.

We can reference graphics, tables, sections, chapters, source code listings, 
and equations. 
Many packages use the label in order to find out the object type of the 
referenced element.
For this reason, it is common to insert a prefix before each label 
(\cref{lst:reference-prefixes}).

\begin{table}[H]
	\centering
	\begin{tabular}{@{}llll@{}}
		\toprule
		Prefix & Object type & Prefix & Object type \\ \midrule
		fig: & figures & tbl:    & tables            \\ 
		sec: & sections  & subsec: & subsections     \\
		ch:  & chapters     & itm:    & items \\
		eq:  & equations & lst:    & source code listings  \\ \bottomrule
	\end{tabular}
	\caption{Prefixes for labels}
	\label{lst:reference-prefixes}
\end{table}

Note that if we use \code{latex}{\textbackslash cref\{<label>\}}\,---\,for some 
document classes\,---\,the generated passages only appear in the desired 
language (e.\,g., German) when the language is specified already within the 
document class command:

\codeblock{latex}{listings/references/set-language.tex}

\noindent Except for sections, captions\footnote{\code{latex}{\textbackslash 
caption\{…\}}} \emph{always need to be specified and positioned before the 
label}\textit{.} Otherwise, they cannot be referenced later on in the text.
Labels for sections are inserted directly after the command:

\codeblock{latex}{listings/references/sections.tex}

