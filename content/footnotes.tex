\chapter{Footnotes}
\label{sec:footnotes}

Whenever we want to include footnotes into our \LaTeX{} document, we can use the command \mintinline{latex}{\footnote{text}}. At the position where we use the command, the correct number will be inserted automatically, and the text within the curly braces will appear in the footer. In combination with the package \mintinline{latex}{hyperref}, URLs within footnotes become clickable\footnote{if we use the command \mintinline{latex}{\url{…}}}. 
We can see examples for that in the whole document.

The package \mintinline{latex}{footmisc} provides us with additional options for how to display footnotes. They can be passed as optional parameters to the command \mintinline{latex}{\usepackage}:
\begin{itemize}
  \item \mintinline{latex}{\usepackage[perpage]{footmisc}} ensures that the count of footnotes begins at 1 for each new page.
  \item \mintinline{latex}{\usepackage[para]{footmisc}} lets the footnotes within the footer appear as continuous text (i.\,e., the footnotes can also appear next to each other).
  \item \mintinline{latex}{\usepackage[symbol]{footmisc}} causes a numbering with symbols (e.\,g.,  \textdagger, \textdaggerdbl) instead of numbers.
\end{itemize}

