\section{Fußnoten}
\label{sec:footnotes}

Für Fußnoten gibt es den Befehl \mintinline{latex}{\footnote{text}}, der an der Stelle des Aufrufs automatisch die richtige Ziffer einfügt und den übergebenen Text in der Fußzeile erscheinen lässt.
In Kombination mit dem Paket \mintinline{latex}{hyperref} sind die Fußnoten sowie URLs\footnote{Sofern sie über den Befehl \mintinline{latex}{\url{…}} gesetzt wurden.} zudem anklickbar.
Beispiele findet ihr im gesamten Dokument.

Das Paket \mintinline{latex}{footmisc} stellt verschiedene weitere Optionen für die Darstellung von Fußnoten zur Verfügung, die als optionale Parameter an den Befehl \mintinline{latex}{\usepackage} übergeben werden können:

\begin{itemize}
  \item \mintinline{latex}{\usepackage[perpage]{footmisc}} sorgt dafür, dass die Zählung der Fußnoten auf jeder Seite neu beginnt.
  \item \mintinline{latex}{\usepackage[para]{footmisc}} lässt die Fußnoten in der Fußzeile als Fließtext (gegebenenfalls auch nebeneinander) erscheinen.
  \item \mintinline{latex}{\usepackage[symbol]{footmisc}} bewirkt eine Nummerierung mit Symbolen (z.\,B. \textdagger, \textdaggerdbl) statt Ziffern.
\end{itemize}

