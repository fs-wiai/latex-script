\chapter{Footnotes}
\label{sec:footnotes}

Whenever we want to include footnotes into our \LaTeX{} document, we can use 
the command \code{latex}{\textbackslash footnote{<text>}}. At the position where we use 
the command, the correct number will be inserted automatically, and the text 
within the curly braces will appear in the footer. In combination with the 
package \pkg{hyperref}, \acro{URL}s within footnotes become 
clickable.\footnote{if we use the command \code{latex}{\textbackslash url{…}}}
We can see examples for that in the whole document.

The package \pkg{footmisc} provides us with additional options for how to display footnotes. They can be passed as optional parameters to the command \code{latex}{\textbackslash usepackage}:
\begin{itemize}
  \item \code{latex}{\textbackslash usepackage[perpage]\{footmisc\}} ensures that the count of footnotes begins at 1 for each new page.
  \item \code{latex}{\textbackslash usepackage[para]\{footmisc\}} lets the footnotes within the footer appear as continuous text (i.\,e., the footnotes can also appear next to each other).
  \item \code{latex}{\textbackslash usepackage[symbol]\{footmisc\}} causes a numbering with symbols (e.\,g.,  \textdagger, \textdaggerdbl) instead of numbers.
\end{itemize}

