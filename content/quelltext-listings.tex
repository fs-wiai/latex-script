\section{Quelltext-Listings}

Um Quelltext in eigenen Arbeiten darzustellen, gibt es in \LaTeX verschiedene Wege.
Wir verwenden hierfür gern das Paket \mintinline{latex}{minted}, das zwar etwas zusätzlichen Installationsaufwand verursacht, dafür aber ansprechende Ergebnisse generiert und gut handzuhaben ist.

\subsection{Installation}
\todo{Python- und Pip-Installation unter allen OS testen}
Für \mintinline{latex}{minted} zu verwenden, ist eine Installation der Programmiersprache Python 3 notwendig.
Die Installationsdateien sind auf der Projekt-Website\footnote{Verfügbar unter \url{https://www.python.org/downloads/}.} zu finden.
Ein ausführlicherer Beitrag zur Installation unter Windows\footnote{Jason FitzpatrickHow to Install Python on Windows. How-To Geek. July 11, 2017. Verfügbar unter \url{https://www.howtogeek.com/197947/how-to-install-python-on-windows/}.} führt euch durch die verschiedenen Schritte, die speziell für dieses Betriebssystem notwendig sind\footnote{Insbesondere das Anpassen des Systempfads sollte nicht vergessen werden.}.

Nach der erfolgreichen Installation solltet ihr in der Eingabeaufforderung\footnote{Navigation unter Windows: \faWindows\ + R → \enquote{cmd} eingeben → Enter} folgende Eingabe vornehmen können:

\begin{minted}[]{bash}
$ python3 --version
Python 3.8.5
\end{minted}

Sollte die Versionsnummer höher sein, ist alles bestens. 
Im gleichen Fenster gebt ihr jetzt den Befehl \mintinline{bash}{pip install Pygments} ein, womit das Pygments-Paket für Python installiert wird.
Ab jetzt könnt ihr das \LaTeX-Paket \mintinline{latex}{minted} über \mintinline{latex}{\usepackage{minted}} verwenden.

\subsection{Compiler-Befehl ändern}
Bevor es allerdings richtig losgehen kann, müssen wir noch eine Kleinigkeit im Compile-Prozess anpassen.
Normalerweise ruft euer Editor beim Klick auf den grünen Pfeil etwa den folgenden Befehl auf: 

\begin{minted}[]{bash}
$ pdflatex main.tex
\end{minted}

Welcher Befehl das bei euch genau ist, könnt ihr in \TeX studio unter \enquote{Optionen → \TeX studio konfigurieren → Befehle} einsehen.
Neben Pdf\LaTeX\ seht ihr den Befehl.
Die Datei, die kompiliert werden soll, wird hier durch \mintinline{bash}{%.tex} angegeben.
Außerdem sind üblicherweise noch zwei zusätzliche Angaben enthalten, jeweils mit einem Minus davor (\mintinline{bash}{-synctex=1 -interaction=nonstopmode}).
Diese Angaben heißen Flags und konfigurieren das Programm \mintinline{bash}{pdflatex}. 
Eine solche Flag müssen wir ergänzen.
Platziert die Angabe \mintinline{bash}{--shell-escape} vor der Datei-Angabe (\mintinline{bash}{%.tex}):

\begin{minted}[]{bash}
pdflatex -synctex=1 -interaction=nonstopmode --shell-escape %.tex
\end{minted}

Nach Klick auf \enquote{Okay} ist der Einrichtungsschritt abgeschlossen.
Auch andere Editoren bieten die Möglichkeit, den Kompilierbefehl anzupassen.
Schaut am besten mal in die Einstellungen oder bemüht eine Suchmaschine.

\subsection{Quelltext setzen}
Jetzt kann der eigentliche Spaß losgehen.
Quelltext könnt ihr von nun an unter Angabe der Sprache in einer eigenen Umgebung setzen:

\example{lst:MintedEnvironment}{code/minted-umgebung}{Beispielhaftes Quelltext-Listing.}

\noindent Außerdem gibt es eine Kurzschreibweise und eine Inline-Variante des Befehls:

\example{lst:MintedVariants}{code/minted-varianten}{Kurzschreibweise und Inline-Listing.}

Um Redundanz zu vermeiden, ist es manchmal praktisch, den Quelltext direkt aus der Quelldatei einzulesen.
Hierzu müssen nur die verwendete Programmiersprache und der Dateipfad an den Befehl \mintinline{latex}{\inputminted} übergeben werden:

\example{lst:MintedExternal}{code/minted-import}{Einbinden aus einer externen Datei.}

\subsection{Minted konfigurieren}

Durch optionale Parameter können Zeilennummerierung, Umbrüche und Farben ausgewählt werden. 
Außerdem sind zahlreiche Themes verfügbar.
Einen umfänglichen Überblick geben die Einführung von Overleaf und die Dokumentation\footnote{Verfügbar unter \url{https://www.overleaf.com/learn/latex/Code_Highlighting_with_minted} bzw. \url{https://ctan.kako-dev.de/macros/latex/contrib/minted/minted.pdf}.}.

\example{lst:MintedExternalStyled}{code/minted-import-styled}{Themes und weitere Optionen.}
