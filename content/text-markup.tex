\chapter{Text markup}

\section*{Text highlighting}

Text markup can be done in two ways: semantically or visually.
We recommend that you use semantic markup whenever possible.
In contrast to visual markup, it only states \emph{why} something is special and entrusts to \LaTeX{} \emph{how} it is going to look.
The simplest semantic markup, that was also used in the previous sentence, is an emphasis: \code{latex}{\textbackslash emph\{…\}}.
By default, this command sets text in italics.
When it is nested, the second level of emphasis is set straight again.
This kind of formatting is only perceived when reading the text and does not attract attention beforehand, as colored or bold text does (which is more appropriate for higher-level structuring purposes).

Some types of visual markup are listed in \cref{tbl:visual-markup}, but you should use them very carefully.
In principle, they can also be nested, however, for some combinations, the corresponding fonts will be missing.
Many other programs try to distort existing fonts to imitate the missing one.
As this does not look particularly good, \LaTeX{} will not do it.
So do not be surprised when your carefully nested selection of four different markups is just ignored and does not do anything at all.

\begin{table}[H]
	\center
	\begin{tabular}{lll}
		\toprule
		Markup & Command & Rendering \\
		\midrule
		bold & \code{latex}{\textbackslash textbf\{bold face\}} & \textbf{bold face} \\
		italics & \code{latex}{\textbackslash textit\{italics\}} & \textit{italics} \\
		small caps & \code{latex}{\textbackslash textsc\{small caps\}} & \textsc{small caps} \\
		monospaced & \code{latex}{\textbackslash texttt\{typewriter text\}} & \texttt{typewriter text} \\
		slanted & \code{latex}{\textbackslash textsl\{slanted\}} & \textsl{slanted} (please, don’t!) \\
		underlined & \code{latex}{\textbackslash underline\{underlined\}} & \underline{underlined} \\
		subscript & \code{latex}{\textbackslash textsubscript\{subscript\}} & x\textsubscript{subscript} \\
		superscript & \code{latex}{\textbackslash textsubscript\{superscript\}} & x\textsuperscript{superscript} \\
		\bottomrule
	\end{tabular}
	\caption{Visual markup commands}
	\label{tbl:visual-markup}
\end{table}

Usually, you should not need these commands too often, as they appear by themselves when you are using semantic markup.
For instance, the \pkg{hyperref} package provides the \code{latex}{\textbackslash url\{…\}} command.
This command does not only use a mono-spaced font for \acro{URL}s, it also makes them clickable and, if necessary, wraps them without adding hyphens.

The same applies for different font sizes.
You can specify the body text font size with an option at the document class:
\codeblock{latex}{listings/text-markup/font-size.tex}
Building upon this, \LaTeX{} generates different font sizes that can be called via the commands in \cref{tbl:type-sizes}.
It is, however, best to restrict those to title pages and similar things.
For the rest, you can trust the default settings and avoid the visual clutter.

\begin{table}[H]
	\center
	\begin{tabular}{ll}
		\toprule
		Command & Rendering \\
		\midrule
		\code{latex}{\{\textbackslash tiny tiny\}} & {\tiny tiny} \\
		\code{latex}{\{\textbackslash footnotesize footnote size\}} & {\footnotesize footnote size} \\
		\code{latex}{\{\textbackslash small small\}} & {\small small} \\
		\code{latex}{\{\textbackslash normalsize normal\}} & {\normalsize normal} \\
		\code{latex}{\{\textbackslash large large\}} & {\large large} \\
		\code{latex}{\{\textbackslash Large larger\}} & {\Large larger} \\
		\code{latex}{\{\textbackslash LARGE largest\}} & {\LARGE largest} \\
		\code{latex}{\{\textbackslash huge largest of all\}} & {\huge largest of all} \\
		\code{latex}{\{\textbackslash Huge megalomania\}} & {\Huge megalomania} \\
		\bottomrule
	\end{tabular}
	\caption{Font size commands}
	\label{tbl:type-sizes}
\end{table}

\newpage

\section*{Paragraph alignment}
\label{sec:ragged-alignment}
By default, \LaTeX{} sets continuous text in full justification. 
However, we can also switch to ragged alignment by using the commands 
\code{latex}{\textbackslash raggedright}, \code{latex}{\textbackslash raggedleft}, and 
\code{latex}{\textbackslash centering}. 
These commands influence the environment that they are used in, e.\,g., the 
\mono{document} environment. Correspondingly, the text within the whole 
document is affected. 
Alternatively, we can use dedicated environments in order to influence the 
formatting
of certain paragraphs (\cref{lst:ragged-alignment}).

\example{lst:ragged-alignment}{formatting-paragraphs/ragged-alignment}{Ragged 
alignment}

\section*{Indentation and spacing}
\label{sec:indents-and-parskips}
Usually, we illustrate a new paragraph by indenting the first line of it 
(\code{latex}{\textbackslash parindent}). 
Alternatively, paragraph spacing, i.\,e., vertical space between paragraphs, 
can be used (\code{latex}{\textbackslash parskip}).
For both variants, there are adjustable parameters:
\codeblock{latex}{listings/formatting-paragraphs/indentation.tex}

\noindent We can use \code{latex}{\textbackslash noindent} to turn off the indentation for only one 
paragraph. 
For the first paragraph after a heading, there is usually no indentation. 


