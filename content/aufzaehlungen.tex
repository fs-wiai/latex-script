\section{Aufzählungen}

Von Haus aus unterstützt \LaTeX\ drei Aufzählungstypen: ungeordnete und geordnete Listen sowie Definitionslisten.
Zu jedem dieser Typen existieren eigene Umgebungen, die aber grundsätzlich gleich aufgebaut sind.

\example{lst:unordered-list}{aufzaehlungen/verschachtelt}{Beispielhafte ungeordnete Aufzählung}

In \cref{lst:unordered-list-code} werden verschiedene Listenelemente (\mintinline{latex}{\item}) von einer \mintinline{latex}{itemize}-Umgebung eingerahmt.
Innerhalb eines Listenelements kann eine neue Listen\-umgebung eröffnet werden, um eine verschachtelte Liste zu erhalten.
Ersetzt man \mintinline{latex}{itemize} durch \mintinline{latex}{enumerate}, erhält man bei sonst gleichem Quelltext eine geordnete Liste.
Für etwas weniger gebräuchliche Definitionslisten muss die Umgebung \mintinline{latex}{description} verwendet werden.
Außerdem erhalten die Elemente in eckigen Klammern den zu definierenden Begriff (\cref{lst:definition-list}).

\Example{lst:definition-list}{aufzaehlungen/definition}{aufzaehlungen/definition_crop}{Beispielhafte Definitionsliste}

Die sehr großen Abstände zwischen einzelnen Listenelementen können durch das Paket \mintinline{latex}{paralist}\footnote{\url{https://www.ctan.org/pkg/paralist}} verringert werden.
Hierzu wird nur der Umgebungsname ersetzt: \mintinline{latex}{compactitem} statt \mintinline{latex}{itemize}, \mintinline{latex}{compactenum} statt \mintinline{latex}{enumerate} und \mintinline{latex}{compactdesc} statt \mintinline{latex}{description}.
Soll die Aufzählung im Fließtext erscheinen, bringt \mintinline{latex}{paralist} dafür die Umgebungen \mintinline{latex}{inparaenum} und \mintinline{latex}{inparaitem} mit.

Um das Aufzählungszeichen oder die Nummerierung anzupassen, kann das Paket \mintinline{latex}{enumitem}\footnote{\url{https://www.ctan.org/pkg/enumitem}} verwendet werden.
\mintinline{latex}{\begin{enumerate}[label=\roman*]} erzeugt eine Liste mit römischen Zahlen. Für alphabetische Nummerierung sorgt der Parameter \mintinline{latex}{[label=\alph*]}.
