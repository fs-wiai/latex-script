\section{Ausblick}

\todo[inline]{Einleitender Absatz}

\subsection{Pakete}

Einige Pakete haben wir euch bereits vorgestellt, es gibt aber noch ein paar tausend weitere.
Für einige häufig benötigte Features haben wir euch hier eine kurze Liste passender Pakete zusammengestellt:\todo{Eventuell ein paar Beispielgrafiken einbinden?}

\begin{description}
	\item[Stichwortverzeichnisse]
		können mit \texttt{makeidx} automatisiert erstellt werden.\footnote{\url{https://www.ctan.org/pkg/makeidx}}
		Mit \mintinline{tex}{\index{…}} werden im Text einzelne Stichwörter ausgezeichnet, \mintinline{tex}{\printindex} sammelt sie in einem Verzeichnis mit Referenzen.
	\item[Vektorgrafiken]
		lassen sich mit \texttt{TikZ} (rekursives Akronym für \emph{TikZ ist kein Zeichenprogramm}) direkt im \LaTeX{}-Code erstellen.\footnote{\url{https://www.ctan.org/pkg/pgf}}
		Achtung: Dieses Paket ist sehr mächtig, aber nicht unbedingt einsteigerfreundlich.
		Bevor ihr damit etwas von Grund auf selbst gestaltet, empfehlen wir euch, mit einigen der Beispiele bei \TeX{}ample\footnote{\url{https://texample.net/tikz/examples/}} zu experimentieren.
		Für bestimmte Anwendungsfälle gibt es aber auch spezielle Pakete, die dann meist einfacher zu handhaben sind:
	\item[Konstituentenbäume,]
		die Sätze in ihre grammatikalischen Bestandteile zerlegen, erzeugt \texttt{qtree}.\footnote{\url{https://ctan.org/pkg/qtree}}
	\item[Beweisbäume,]
		wie sie in der Logik benötigt werden, erzeugt \texttt{prftree}.\footnote{\url{https://www.ctan.org/pkg/prftree}}
	\item[Chemische Strukturformeln]
		können unter anderem mit \texttt{chemfig} erzeugt werden.\footnote{\url{https://www.ctan.org/pkg/chemfig}}
	\item[Farbe]
		bringt \texttt{xcolor} in eure Dokumente.\footnote{\url{https://www.ctan.org/pkg/xcolor}}
	\item[Notizen,]
		die ihr bei der Abgabe garantiert nicht überseht, fügt \texttt{todonotes} ein.\footnote{\url{https://www.ctan.org/pkg/todonotes}}
		Damit könnt ihr markieren, was ihr noch ändern oder einfügen wollt.
	\item[Seiten aus anderen \acro{PDF}-Dateien]
		integriert ihr mit \texttt{pdfpages}.\footnote{\url{https://www.ctan.org/pkg/pdfpages}}
		Das eignet sich sehr gut, um Ausgaben anderer Programme in eure Arbeit zu integrieren, beispielsweise in einem Anhang. 
		Einmal kompilieren, und schon ist auch der Anhang wieder auf dem neuesten Stand, wenn das externe Programm etwas geändert hat.
	\item[Verschachtelte Abbildungen]
		und die nahezu beliebige Positionierung von Bildunterschriften ermöglicht \texttt{subcaption}.\footnote{\url{https://www.ctan.org/pkg/subcaption}}
		Davon haben wir auch in diesem Dokument ausgiebig Gebrauch gemacht.
	\item[Tabellen]
		können noch sehr viel flexibler gestaltet werden, als wir es hier gezeigt haben.
		Dabei helfen unter anderem die Pakete 
		\todo{War da die Länge des Namens das entscheidende Auswahlkriterium? :D}
		\texttt{colortbl},\footnote{\url{https://www.ctan.org/pkg/colortbl}}
		\texttt{tabularx},\footnote{\url{https://www.ctan.org/pkg/tabularx}}
		\texttt{multirow},\footnote{\url{https://www.ctan.org/pkg/multirow}}
		\texttt{makecell}.\footnote{\url{https://www.ctan.org/pkg/makecell}}
\end{description}

\noindent Eigentlich kein Paket, sondern eine weitere Dokumentenklasse ist \textbf{beamer:} Damit könnt ihr \textbf{Bildschirmpräsentationen} mit \LaTeX erstellen.
Informationen und Beispiele dazu gibt es bei Overleaf\footnote{\url{https://www.overleaf.com/learn/latex/Beamer}} –
womit wir schon beim nächsten Abschnitt sind:

\subsection{Hilfe und Informationen}
