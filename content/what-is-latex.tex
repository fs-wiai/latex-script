\chapter{Was ist \LaTeX?}
\label{sec:what-is-latex}

Anfang der 1960er-Jahre bekam ein ziemlich begabter amerikanischer PhD-Student den Auftrag, ein Buch über Compiler zu schreiben.
Nachdem er mit der Arbeit begonnen hatte, fragte er seinen Verleger, ob das Buch ein bisschen länger werden dürfe, da er auch die Grundlagen vorstellen wolle.
Der Verleger antwortete, er solle das Buch so lange machen, wie er es für nötig halte.
1968 erschien der erste Band, damals noch in Bleisatz gedruckt.

Der Bleisatz wurde zu dieser Zeit durch andere Verfahren verdrängt, mit deren Qualität der Autor aber nicht zufrieden war. 
Er begann deshalb Ende der 70er-Jahre mit der Entwicklung seines eigenen Textsatzsystems \TeX, \emph{tech} ausgesprochen – nach dem altgriechischen Wort \texttau$\mathrm{\acute{\varepsilon}}$\textchi\textnu\texteta{} (technē) für \emph{Kunst}.

Inzwischen ist Donald Knuth (so heißt der ehemalige Student) emeritierter Informatikprofessor und sein Compilerbuch auf das mehrbändige Standardwerk \emph{The Art of Computer Programming} angewachsen – von dem allerdings drei Bände noch ausstehen, darunter auch der über Compiler.
Im Gegensatz zum Buch ist \TeX{} heute die seltene Erscheinung eines Softwaresystems, das man als \emph{fertig} bezeichnen darf, ohne damit \emph{tot} zu meinen.

Zwei Buchstaben fehlen noch zu \LaTeX: 
Es sind die Anfangsbuchstaben des Nachnamens von Leslie Lamport, der in den 80er-Jahren basierend auf \TeX{} eine Sammlung von kleinen Programmen geschrieben hat, die das System für uns Endanwender nutzbar machten und so zu seiner Verbreitung beitrugen.
Die aktuelle Version \LaTeXe{} gibt es seit Mitte der 90er.

Wozu erzählen wir euch das? 
Weil es einige der Vorteile begründet, die \LaTeX{} heute noch auszeichnen:
Es ist ein ausgereiftes, stabiles, verlässliches System,
das weitgehend automatisch typografisch hochwertigen Textsatz erzeugt.

Da der \TeX-Code in Plaintext-Dateien abgelegt wird (siehe \cref{sec:basic-functionality}), ergeben sich weitere Vorteile:
Ihr könnt eure Projekte übersichtlich strukturieren (\cref{sec:project-structure}),
und wann immer ihr Änderungen im Quelltext rückgängig macht, könnt ihr euch darauf verlassen, wieder die gleiche Ausgabe wie vorher zu erhalten. 
Das klappt in größerem Maßstab natürlich auch in Verbindung mit Git oder anderen Quelltextversionierungstools.
Außerdem bleibt euer Quelltext langfristig zuverlässig lesbar, ohne dass ihr bestimmte Programme dafür benötigt.
Gleichzeitig ist der Austausch mit allen Programmen, die Plaintext unterstützen, sehr einfach.

% Quellen:
% https://en.wikipedia.org/wiki/Donald_Knuth
% https://de.wikipedia.org/wiki/Donald_E._Knuth
% https://en.wikipedia.org/wiki/The_Art_of_Computer_Programming
% https://de.wikipedia.org/wiki/The_Art_of_Computer_Programming
% https://en.wikipedia.org/wiki/TeX
% https://de.wikipedia.org/wiki/TeX
% https://de.wikipedia.org/wiki/LaTeX
% https://en.wikipedia.org/wiki/LaTeX

