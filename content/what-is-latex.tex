\chapter{\replaced[id=C]{Why use}{What is} \LaTeX?}
\label{sec:what-is-latex}

In the early 1960s, a rather talented American Ph.D. student was asked by a big publishing company whether he wanted to write a book on compilers.
He did.
Soon after he had begun with his research, he realized that he wanted to start with some foundations of computer science, so he asked the publishers if the book might be a little longer.
They replied, \enquote{make it as long as you feel necessary.}
In 1968, the first volume was published, at that time still printed using mechanical typesetting.

This method was just disappearing then, and being replaced by new methods.
However, the author did not like the results of those new methods, so,
at the end of the 70s, he began to develop his own typesetting system \TeX{}
(pronounced as \emph{tech}\todo{is that a valid transliteration in english?}), named after the ancient Greek word \texttau$\mathrm{\acute{\varepsilon}}$\textchi\textnu\texteta{} (technē) meaning \emph{art, craft}.

Today, Donald Knuth (that is the former student’s name) is a retired professor of computer science and his compiler book has grown to become the multi-volume standard work \emph{The Art of Computer Programming}\,—\,three volumes of which are still to be written, among them the one on compilers.
Unlike the book, however, \TeX{} is the rare occurrence of a software system that may actually be called \emph{complete} without meaning \emph{dead}.

Two more letters are needed for \LaTeX:
They are the initial two letters of the last name of Leslie Lamport, who, in the 80s, extended \TeX{} by a collection of small programs that made the entire system usable for us end-users and are responsible for its widespread adoption.
The current version dates from the mid~90s.

Why are we telling you all of that?
It explains some of the advantages that still distinguish \LaTeX{} today:
It is a mature, stable, and reliable system
that does typesetting in a typographically sophisticated way and mostly automatically.

As the \TeX{} code is stored in plaintext files (cf. \cref{sec:basic-functionality}),
even more advantages arise:
You can structure your projects clearly (cf. \cref{sec:project-structure}),
and whenever you undo changes in the source code, you can always rely on getting exactly the same output as before
\added[id=C]{rather than some more or less similar reconstruction}.
\todo{Klingt für mich, als würden Undos nichts bewirken. (F)}
On a larger scale, this does also work in connection with Git or other source code versioning tools.
Furthermore, you can trust your source code to be readable long-term, without any specific software.
It can always be opened with any program that supports plaintext.

% Quellen:
% https://en.wikipedia.org/wiki/Donald_Knuth
% https://de.wikipedia.org/wiki/Donald_E._Knuth
% https://en.wikipedia.org/wiki/The_Art_of_Computer_Programming
% https://de.wikipedia.org/wiki/The_Art_of_Computer_Programming
% https://en.wikipedia.org/wiki/TeX
% https://de.wikipedia.org/wiki/TeX
% https://de.wikipedia.org/wiki/LaTeX
% https://en.wikipedia.org/wiki/LaTeX

