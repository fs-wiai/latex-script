\section{Tabellen}

Für Tabellen gibt es zwei grundlegende Umgebungen.
Die erste, \texttt{table}, ist dafür zuständig, die Tabelle insgesamt in das übrige Dokument einzubauen.
Die Positionierung erfolgt dabei analog zur Positionierung von Grafiken.
Auch bei Tabellen kann der Befehl \mintinline{tex}{\caption} verwendet werden, um der Tabelle eine Beschriftung zuzuordnen.

Anders als bei Grafiken funktioniert dann der eigentliche Tabelleninhalt.
Während Grafiken aus externen Dateien stammen und nicht durch \LaTeX interpretiert werden, muss die innere Struktur von Tabellen im Code aufgeschlüsselt werden.
Dazu dient die Umgebung \texttt{tabular}, die als verpflichtenden Parameter eine Spaltendefinition erwartet.
Die Spaltendefinition besteht aus je einem Buchstaben pro Tabellenspalte, der die Textausrichtung der Spalte angibt: 
\texttt{l} für linksbündig, \texttt{r} für rechtsbündig, \texttt{c} für zentriert.

Innerhalb der \texttt{tabular}-Umgebung folgt dann der eigentliche Tabelleninhalt.
Dabei werden Tabellenzeilen genau wie Zeilenumbrüche durch \mintinline{tex}{\\} markiert, Zellengrenzen durch \mintinline{tex}{&}.

Für typografisch ansprechende Tabellen empfehlen wir das Paket \texttt{booktabs}.\footnote{Alle bisher genannten Befehle funktionieren auch ohne dieses Paket, das Ergebnis sieht dann aber deutlich unprofessioneller aus.}
Dieses Paket bringt unter anderem die Befehle \mintinline{tex}{\toprule}, \mintinline{tex}{\midrule} und \mintinline{tex}{\bottomrule} mit, die jeweils passende horizontale Linien für den Tabellenbegin, für Gruppierungen innerhalb der Tabelle sowie für das Tabellenende zeichnen.

Vertikale Linien können als senkrechter Strich (\texttt{|}) in die Spaltendefinition eingefügt werden, davon ist aber abzuraten.
Wenn ihr den zusätzlichen Weißraum entfernen möchtet, der standardmäßig die Spalten umgibt, könnt ihr – ebenfalls zwischen den Buchstaben der Spaltendefinition – die Zeichenkette \mintinline{tex}{@{}} einfügen.

Die komplette Tabelle kann dann beispielsweise so aussehen:

\todo[inline]{Beispieltabelle (Code und Darstellung)}

\paragraph{Überlänge}
Für Tabellen, die über eine Seite hinausgehen, oder die Zeilenumbrüche innerhalb einzelner Tabellenzellen erfordern, kann zusätzlich das Paket \texttt{longtable} eingebunden werden, mit dem \texttt{booktabs} ebenfalls kompatibel ist.
Die \texttt{longtable}-Umgebung vereint die \texttt{table}- und die \texttt{tabular}-Umgebung.
Damit ergibt sich folgende Grundstruktur:

\todo[inline]{Beispieltabelle (nur Code)}

\paragraph{Überbreite}
Soll stattdessen eine sehr breite Tabelle dargestellt werden, empfiehlt es sich auf hochkant ausgerichteten Seiten, die Tabelle um 90\textdegree{} zu drehen.
Das geht mithilfe des Pakets \texttt{rotating}.
Der einzige Unterschied zu einer normalen Tabelle ist, dass die \texttt{table}- Umgebung durch eine \texttt{sidewaystable}-Umgebung ersetzt wird.
Die Positionierung und die enthaltene \texttt{tabular}-Umgebung funktionieren unverändert.

\paragraph{Weitere Möglichkeiten}
Natürlich bietet \LaTeX{} noch viele weitere Features für ausgefeiltere Tabellen, beispielsweise zeilen- oder spaltenübergreifende Zellen. 
Dafür, oder wenn ihr euch einfach Tipparbeit sparen wollt, empfehlen wir euch den \emph{Tables Generator,}\footnote{\url{https://tablesgenerator.com/}} in dem ihr Tabellen in \textsc{wysiwyg}-Manier zusammenklicken könnt und kopierfertigen \LaTeX-Code erhaltet.


