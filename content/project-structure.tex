\section{Projektstruktur}
\label{sec:project-structure}

In den bisherigen Kapiteln haben wir nur sehr kurze \LaTeX{}-Beispiele gesehen. Natürlich kann \LaTeX{} aber auch verwendet werden, um größere Dokumente und Projekte zu erstellen, wie etwa eine Abschlussarbeit. 
Um nicht den Überblick im Quellcode zu verlieren und zu vermeiden, dass Quelldateien zu lang werden, ist eine sinnvolle Strukturierung eines größeren \LaTeX{}-Projektes ratsam. Hierfür wird der Quellcode in verschiedene Dateien aufgeteilt, welche in den folgenden Abschnitten näher beleuchtet werden.

\subsection{Hauptdatei} Typischerweise wird bei größeren Projekten mit einer Hauptdatei gearbeitet, die gerne \texttt{main.tex} genannt wird. Sie ist sozusagen das Skelett des Projektes und enthält damit dessen Grundgerüst inklusive der Präambel. In dieser Hauptdatei werden Titel, Inhaltsverzeichnis sowie die einzelnen Kapitel einer Arbeit eingebunden (vgl. \cref{lst:main-file}). Die Einbindung der einzelnen Abschnitte kann entweder mittels \mintinline{latex}{\input{...}} oder \mintinline{latex}{\include{...}} erfolgen. Beide verlangen als Argument den Pfad zur Datei, die eingebunden werden soll. Auf die Unterschiede zwischen den beiden Befehlen gehen wir später noch ein (vgl. \cref{subsec:input-vs-include}).
  
\example{lst:main-file}{project-structure/main-file}{Typische Struktur einer Hauptdatei in \LaTeX{}}
 
\subsection{Abschnittsdateien}
 Abschnittsdateien sind Dateien, die innerhalb der Hauptdatei eingebunden werden. Diese können in einer Abschlussarbeit beispielsweise einzelne Kapitel sein. Grundsätzlich ist man aber ganz frei in der Entscheidung, wie granular die Aufteilung des Inhaltes in einzelne Dateien gestaltet werden soll.
 Die Dateien, die durch die Hauptdatei eingebunden werden, enthalten keine Präambel, da diese ja schon in der Hauptdatei vorhanden ist. Auch die Befehle \mintinline{latex}{\begin{document}} und \mintinline{latex}{\end{document}} tauchen nicht noch mal auf.
 
\subsection{Input oder Include?}
\label{subsec:input-vs-include}
Kommen wir nun zu den Unterschieden zwischen den beiden möglichen Befehlen zum Einbinden von \LaTeX-Dateien. Kurz zusammengefasst weisen \mintinline{latex}{\include} und \mintinline{latex}{\input} Unterschiede in den folgenden drei Aspekten auf: Die Art und Weise, wie der Pfad zur einzubindenden Datei angegeben wird, die Möglichkeit der Schachtelung von Einbindungen und ob für jeden Abschnitt eine neue Seite angefangen wird.
 
Benutzt man den Befehl \mintinline{latex}{\input}, kann die Dateiendung \texttt{.tex} angegeben werden, sie ist aber nicht zwingend notwendig. Außerdem kann man die Einbindung von Dateien schachteln: Eine Datei, die mit \mintinline{latex}{\input} eingebunden wurde, kann wiederum mit dem gleichen Befehl eine andere Datei einbinden. Die Dateien, die eingebunden wurden, werden im fertigen Dokument eingefügt, ohne dass dabei eine neue Seite für den eingebundenen Abschnitt angefangen wird (vgl. \cref{lst:main-file}).
 
Anders verhält sich der Befehl \mintinline{latex}{\include}: Hier wird die Dateiendung \texttt{.tex} für die eingebundenen Kapitel \emph{nicht} mit angegeben. Die Schachtelung von Einbindungen ist nicht möglich. Für jede eingebundene Datei wird außerdem eine neue Seite erzeugt. 