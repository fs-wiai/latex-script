\section{Absatzformatierung} % (fold)
\label{sec:absatzformatierung}

\subsection*{Flattersatz} % (fold)
\label{sub:flattersatz}
Standardmäßig setzt \LaTeX \ Fließtext im Blocksatz. Flattersatz ist aber auch möglich. Die Befehle \texttt{\textbackslash raggedright}, \texttt{\textbackslash raggedleft}, und \texttt{\textbackslash centering} können dazu verwendet werden. Diese Befehle beeinflussen die Umgebung, in der sie verwendet werden. Dies kann zum Beispiel die \texttt{document}-Umgebung sein, dementsprechend würde sämtlicher Text des Dokuments beeinflusst werden.
Alternativ existieren Umgebungen, welche die Absatzformatierung beeinflussen:
\begin{verbatim}
	\begin{flushleft} TEXT \end{flushleft}
	\begin{flushright} TEXT \end{flushright}
	\begin{center} TEXT \end{center}
\end{verbatim}
\begin{flushleft}
	Dieser Text steht in einer \texttt{flushleft}-Umgebung. Lorem ipsum dolor sit amet, consectetur adipisicing elit, sed do eiusmod tempor incididunt ut labore et dolore magna aliqua.
\end{flushleft}
\begin{flushright}
	Dieser Text steht in einer \texttt{flushright}-Umgebung. Lorem ipsum dolor sit amet, consectetur adipisicing elit, sed do eiusmod tempor incididunt ut labore et dolore magna aliqua.
\end{flushright}
\begin{center}
	Dieser Text steht in einer \texttt{center}-Umgebung. Lorem ipsum dolor sit amet, consectetur adipisicing elit, sed do eiusmod tempor incididunt ut labore et dolore magna aliqua.
\end{center}
% subsection flattersatz (end)
\subsection*{Einzüge und Abstände} % (fold)
\label{sub:einzüge_und_abstände}
Absätze werden standardmäßig durch Einzüge in der ersten Zeile des Absatzes verdeutlicht (\texttt{\textbackslash parindent}). Stattdessen können auch vertikale Abstände zwischen den Absätzen verwendet werden (\texttt{\textbackslash parskip}). Für beide Varianten gibt es einstellbare Parameter:
\begin{verbatim}
\setlength{\parindent}{0pt}
\setlength{\parskip}{1em
    plus .5em   % erlaubte Dehnung
    minus .5em  % erlaubte Stauchung
}
\end{verbatim}
Mit dem Befehl \texttt{\textbackslash noindent} kann für nur einen Absatz der Einzug abgeschaltet werden. Für den ersten Absatz nach einer Überschrift wird standardmäßig automatisch kein Einzug eingefügt.
% subsection einzüge_und_abstände (end)
