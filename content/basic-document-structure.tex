\chapter{Basic Document Structure}
\label{sec:basic-document-structure}

How does a \LaTeX{} document look like?
In essence, every \LaTeX{} document is composed of two parts: the first part is 
the preamble which is followed by the second part, the document environment.

We call the first commands within our \LaTeX{} document a \emph{preamble}. It 
contains global information about our document, such as the document class that 
we want to use, the encoding, the language, the page format, and additional 
packages that we use. 
The \emph{document environment}, on the other hand, contains the actual content 
of our document, that is, the things that we will later see in our generated 
\acro{PDF} file. 

\Example{lst:latex-document-basic-structure}{basic-document-structure/hello-world}{basic-document-structure/hello-world_crop}{Beispielhafter
	Structure of a simple \LaTeX{} document with preamble and document 
	environment}

\section{Preamble}
Let's take a closer look at the preamble. 
A minimal preamble should contain the following specifications: 

\subsection{Document Class}\label{sec:document-class}
We can define a document class by using the command 
\mintinline{latex}|\documentclass[<parameter>]{<document class>}|. The most 
commonly used document classes that are supported by default are 
\mintinline{latex}{article} for short documents, and \mintinline{latex}{report} 
for longer ones. Futhermore you can use \mintinline{latex}{book} for books, 
\mintinline{latex}{beamer}\footnote{We do not cover making presentations in 
	\LaTeX{} in this tutorial. However, if you are interested in the topic, we 
	recommend this introduction on Overleaf: 
	\url{https://www.overleaf.com/learn/latex/Beamer}} for presentations, and 
\mintinline{latex}{letter}\footnote{We also do not cover letters in this 
	script. An introduction can be found on WikiBooks: 
	\url{https://en.wikibooks.org/wiki/LaTeX/Letters}} for letters.

In addition to the standard document classes, the \acro{KOMA} script classes 
developed over time. It offers alternatives to the document classes mentioned 
above: In lieu of \mintinline{latex}{article} you can use 
\mintinline{latex}{scrartcl}, \mintinline{latex}{report} is replaced by 
\mintinline{latex}{scrreport}, and \mintinline{latex}{scrbook} can be used 
instead of \mintinline{latex}{book}. As a replacement for 
\mintinline{latex}{letter} one can use \mintinline{latex}{scrlttr2}. 
A complete list of all \acro{KOMA} script classes is available 
online\footnote{Erreichbar unter 
	\url{https://komascript.de/komascriptbestandteile}}. By using \acro{KOMA} 
document classes, the layout of the generated \acro{PDF} document is changed. 
On top of that, they provide additional functionality. 
The standard document classes are designed according to the American-English 
standards whereas \acro{KOMA} classes adhere to European norms, such as for 
writing letters. 

Each \mintinline{latex}{\documentclass} command can hold optional parameters in 
square brackets. 
\mintinline{latex}{\documentclass[10pt,a5paper,landscape]{scrartcl}}, for 
instance, configures a \acro{KOMA} script article ans specifies its font size 
to 10\,pt\footnote{The standard font size is 12,pt.}, sets the page size to 
A5\footnote{The default case would be A4} and the orientation of the page to 
landscape. The language can be passed as an optional parameter, too (cf. 
\cref{sec:language}). 

\subsection{Digression: Packages}
\label{sec:packages}
\begin{minted}{latex}
\usepackage[<optionen>]{<paketname>}
\end{minted}
Packages provide additional commands and functionalities that we can use within 
our \LaTeX{} source code. There are numerous packages for different use cases 
(e.\,g. typesetting forumlas, lists, \textellipsis). 
In order make use of a package, it must be included within the preamble. To do 
so, the above-mentioned command is used. 
The most important \LaTeX{} packages can be found n the Comprehensive \TeX\ 
Archive Network,\footnote{Available at: \url{https://www.ctan.org/}}, short: 
\acro{CTAN}. You can also find the documentations for the packages there. 

\subsection{Encoding}
\begin{minted}{latex}
\usepackage[utf8]{inputenc}
\usepackage[t1]
\end{minted}

One use case for packages is specifying the encoding of our \LaTeX{} document. 
The character encoding\footnote{Vgl. 
	\url{https://en.wikipedia.org/wiki/Character_encoding}} determines the 
available character set.
The standard encoding in \LaTeX{} is \acro{ASCII}. \footnote{cf. 
	\url{https://de.wikipedia.org/wiki/American_Standard_Code_for_Information_Interchange}}
It is an American character encoding and therefore does, for instance, not 
contain German umlauts, or other special characters, which makes it unsuitable 
for most use cases. 
As a consequence, \acro{UTF-8}footnote{cf. 
	\url{https://en.wikipedia.org/wiki/UTF-8}} can be used as a universal 
	character 
encoding.

In \LaTeX{} we need to specify two character encodings:
The input encoding (short: \mintinline{latex}{inputenc})) which refers to our 
source code. The font encoding (short: \mintinline{latex}{fontenc}) concerncs 
the choice of the file that is used to portray the content of our document 
since we also need special characters there.\footnote{Details  on how 
	\mintinline{latex}{fontenc} works can be found at: 
	\url{https://www.texwelt.de/fragen/5537/was-macht-eigentlich-usepackaget1fontenc}}
\mintinline{latex}{T1} is an enconding that tries to cover most European 
language with a limited number of characters. 

\subsection{Language}\label{sec:language}
\begin{minted}{latex}
\usepackage[ngerman]{babel}
\end{minted}

The package \mintinline{latex}{babel} provides language-specific information 
(e.\,g. on hyphenation, special characters, changing fonts, translation of 
labels\footnote{cf. \cref{sec:references}} like \enquote{Chapter}, 
\enquote{Table of Contents}, or \enquote{Figure}).
The language that you want to use can be passed as an optional parameter. 
Als optionaler Parameter kann die Sprache übergeben werden. 
\mintinline{latex}{ngerman}, for instance, is used for the new German spelling. 
Some packages require that the language is already passed as optional parameter 
in the \mint{latex}{\documentclass} command. In this case, just leave aus the 
optional parameter for the language within the \mintinline{latex}{babel} 
command.

You can use multiple languages in your document. To do so, pass the languages, 
separated by commas, as optional parameter to the \mintinline{latex}{\babel} 
command. Within your document, you can switch between langauges with the 
\mintinline{latex}{\selectlanguage{<language>}} command. Alternatively, you can 
include foreign-language text by using the following command: 
\begin{minted}{latex}
\foreignlanguage{<sprache>}{<text>}
\end{minted}

\section{Document Environment}
The actual content of your \acro{PDF} document needs to be put between 
\mintinline{latex}{\begin{document}} and \mintinline{latex}{\end{document}}.

\subsection{Continuous Text}
The easiest that you can integrate into the document environment is continuous 
text. You can directly write it into your source code. Line breaks and multiple 
spaces are ignored by \LaTeX{}. Blank lines create a new paragraph, which are 
indented by default. \footnote{The automatic indentation of new paragraphs can 
be prevented by using the command \mintinline{latex}{\noindent}.}
Manual linebreaks can be forced with two backslashes 
(\textbackslash\textbackslash). This should be avoided, though. 

\subsection{Comments}
Some characters are reserved for \LaTeX-specific commands. By using a percent 
sign, for instance, the rest of the line is going to be ignroed by the 
compiler, i.\,e. the text will not appear in your generated \acro{PDF} document.
This can be useful in order to take notes while working on your document 
without affecting the actual document. This is called a comment.

However, if you want the percent sign to actually appear in your text, you can 
achieve this by using a backslash: \mintinline{latex}{\%}.
This solution is also called escaping and also works for other reserved 
characters, like \#, \$, \&, \_, \{ and \}.

In order to escape the backslash, the command 
\mintinline{latex}{\textbackslash} must be used.\footnote{An overview of 
additional special characters can be found in \cref{sec:special-characters}.}

\subsection{Sections and Chapters}
Continuous text can be structured by headings that divide the document into 
sections and chapters. Needless to say, \LaTeX{} provides us with commands for 
that.
The commands that are depicted in \cref{lst:headlines} can be used with any 
document class. 
\Example{lst:headlines}{basic-document-structure/headlines}{basic-document-structure/headlines_crop}{Überschriftenebenen}
Depending on your specified document class the commands 
\mintinline{latex}{\chapter{Chapter}} and \mintinline{latex}{\part{Part}} are 
additionally available -- for instance in books. 
You can mark the command with an asterisk if you want to omit the numbering of 
a section and exclude it from the table of contents\footnote{cf. 
\cref{sec:table}}:

\begin{minted}{latex}
\section*{This section is excluded from the table of contents}
\end{minted}

An alternative title for the table of contents can be declared as an optional 
parameter in square brackets between the command and the actual title.

\begin{minted}{latex}
\section[Title in the TOC]{Actual Chapter Title}
\end{minted}

\subsection{Front Matter}
A simple font matter can be created by using the command 
\mintinline{latex}{\maketitle}. The values that get inserted into the front 
matter must be specified within the preamble. 
Multiple authors are joined by \mintinline{latex}{\and}.
Ein einfacher Titel lässt sich im Dokument mit dem Befehl 
\mintinline{latex}{\maketitle} erzeugen.
If the date is not specified by \mintinline{latex}{\date}, the current date 
will be inserted by default.
The design of the front matter depends on the specified document class.

\Example{lst:titles}{basic-document-structure/titles}{basic-document-structure/titles_crop}{Die
 Titelei}

\subsection{Indices}\label{sec:table-of-contents}
If you structure your document with the above-mentioned commands for headings, 
the command \mint{latex}{\tableofcontents} generates a automatically numbered 
table of contents of it (like in \cref{lst:main-file} on 
\cpageref{lst:main-file}).

The numbering style, the depth of the numbering and many other options can, of 
course, be adapted. \footnote{We recommend the following blogpost:
	\url{https://texblog.org/2011/09/09/10-ways-to-customize-tocloflot/}}
For \LaTeX{} to create your table of contents, the project has to be compiled 
twice.

Besides the table of contents, you can also generate a 
\mintinline{latex}{\listoffigures} (list of figures) and a 
\mintinline{latex}{\listoftables} (list of tables). The captions of your 
figures and tables will appear within those indices.\footnote{cf. 
	 \cref{sec:graphics} (Grafiken) and \cref{sec:tables} (Tabellen) for more 
	 information on this.}
