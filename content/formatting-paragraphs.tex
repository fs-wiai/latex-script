\section{Absatzformatierung} % (fold)
\label{sec:formatting-paragraphs}

\subsection*{Flattersatz}
\label{sub:ragged-alignment}
Standardmäßig setzt \LaTeX{} Fließtext im Blocksatz. 
Flattersatz\footnote{Bekannt als links, rechts oder zentriert ausgerichteter Text in Programmen wie Word.} ist aber auch möglich. 
Die Befehle \mintinline{tex}|\raggedright|, \mintinline{tex}|\raggedleft| und \mintinline{tex}|\centering| können dazu verwendet werden. 
Diese Befehle beeinflussen die Umgebung, in der sie verwendet werden. 
Dies kann zum Beispiel die \texttt{document}-Umgebung sein, dementsprechend würde sämtlicher Text des Dokuments beeinflusst werden.
Alternativ existieren Umgebungen, welche die Absatzformatierung beeinflussen (\cref{lst:ragged-alignment}).

\example{lst:ragged-alignment}{formatting-paragraphs/ragged-alignment}{Flattersatz}

\subsection*{Einzüge und Abstände}
\label{sub:indents-and-parskips}
Absätze werden standardmäßig durch Einzüge in der ersten Zeile des Absatzes verdeutlicht (\mintinline{tex}{\parindent}). 
Stattdessen können auch vertikale Abstände zwischen den Absätzen verwendet werden (\mintinline{tex}{\parskip}). 
Für beide Varianten gibt es einstellbare Parameter:
\begin{minted}{tex}
\setlength{\parindent}{0pt}
\setlength{\parskip}{1em
    plus .5em   % erlaubte Dehnung
    minus .5em  % erlaubte Stauchung
}
\end{minted}
Mit dem Befehl \mintinline{tex}{\noindent} kann für nur einen Absatz der Einzug abgeschaltet werden. 
Für den ersten Absatz nach einer Überschrift wird standardmäßig automatisch kein Einzug eingefügt.

