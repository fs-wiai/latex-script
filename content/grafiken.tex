\section{Grafiken einbinden} % (fold)
\label{sec:grafiken_einbinden}
Aufgrund der Natur (What you see is what you mean) von \LaTeX \  stellt die Einbindung von Grafiken in ein Dokument eine interessante Aufgabe dar. Mit textuellen Befehlen zum dargestellten Bild -- wie üblich kommen bestimmte Befehle und Pakete zum Einsatz. In diesem Abschnitt werden diese Bestandteile näher erläutert.

\subsection{Grafiken abbilden} % (fold)
\label{sub:grafiken_abbilden}
Um Grafiken darzustellen, muss das Paket \texttt{graphicx} eingebunden werden. Zum Einfügen können dann folgende Befehle verwendet werden:  

\begin{minted}[tabsize=4]{latex}
\begin{figure}
	\includegraphics{<dateipfad>}
	\caption[<kurztitel>]{<bildunterschrift>}
\end{figure}
\end{minted}

\noindent Soll die Bildgröße angepasst werden, kann der \texttt{includegraphics}-Befehl angepasst werden. 
Die erwünschte Höhe und Breite können hier separat angegeben werden, etwa wie folgt: 

\begin{minted}{latex}
\includegraphics[width=0.5\textwidth,height=5cm]{<dateipfad>}
\end{minted}

% subsection grafiken_abbilden (end)
\subsection{Platzierung} % (fold)
\label{sub:platzierung}
Ein interessanter Aspekt des What-you-see-is-what-you-mean-Paradigmas ist die Art und Weise, wie Grafiken positioniert werden können. Standardmäßig erfolgt eine automatische Platzierung. Dies bedeutet, dass eine Grafik nicht notwendigerweise zwischen den zwei Textblöcken wie im Quelltext erscheint, sondern an anderer Stelle. Soll die automatische Platzierung angepasst werden, kann der \texttt{begin}-Befehl angepasst werden:

\begin{minted}{latex}
\begin{figure}[<positionskürzel>]
\end{minted}

\noindent Die verfügbaren Positionskürzel werden in \cref{tbl:positionskuerzel} aufgelistet.

\begin{table}[h!]
	\center
	\begin{tabular}{cl}
		\toprule
		Kürzel & Position                              \\
		\midrule
		h      & hier, wenn es dir recht ist           \\
		t      & oberer Seitenrand \emph{(top)}        \\
		b      & unterer Seitenrand \emph{(bottom)}    \\
		p      & auf einer eigenen Seite \emph{(page)} \\
		H      & Hier, verdammt noch mal! (benötigt Paket \texttt{float}) \\
		\bottomrule
	\end{tabular}
	\caption{Kürzel zur Platzierung von Abbildungen}
	\label{tbl:positionskuerzel}
\end{table}

Neben der vertikalen Positionierung spielt gegebenenfalls auch die horizontale Ausrichtung eine Rolle. Standardmäßig sind Grafiken linksbündig orientiert. Für eine zentrierte Grafik kann eine \texttt{center}-Umgebung verwendet werden:

\begin{minted}[tabsize=4]{latex}
\begin{figure}[<position>]
	\begin{center}
    	\includegraphics{<dateipfad>}
	\end{center}
\end{figure}
\end{minted}

\noindent Alternativ kann auch der Befehl \mintinline{tex}{\centering} verwendet werden:\todo{Eines streichen? Die beiden unterscheiden sich ja eigentlich nicht …}

\begin{minted}[tabsize=4]{latex}
\begin{figure}[<position>]
	\centering
	\includegraphics{<dateipfad>}
\end{figure}
\end{minted}

\todo{Kapitel mit abschließenden Sätzen beenden?}

