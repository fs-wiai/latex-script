\chapter{Prospects}
\label{sec:prospects}

Obviously, in this script we were not able to show you the least of what \LaTeX{} has to offer. 
Therefore, in this last section, we gathered some information to help you to go further into depth by yourself.

\section{Packages}

We already have presented a selection of packages. However, there are thousands more of them. In the following sections we have put together some packages for frequently needed features: 

\begin{figure}[p]
	\widebox{
		% Top rules:
		\colrules
		% Left content: code listing:
		\begin{subfigure}{\widefigurewidth}
			\includegraphics[width=\linewidth]{graphics/coffee-cup.pdf}
		\end{subfigure}
		\hspace{\widefiguregap}
		% Right content: image or rendered example:
		\begin{subfigure}{\widefigurewidth}
			\includegraphics[width=\linewidth]{graphics/qtree.png}
		\end{subfigure}
		% Bottom rules:
		\colrules
		% Left caption:
		\begin{subfigure}[t]{\widefigurewidth}
			\caption{Vector graphics with TikZ}
			\centering\tiny{\url{https://texample.net/tikz/examples/coffee-cup/}}
			\label{fig:tikz-example}
		\end{subfigure}
		\hspace{\widefiguregap}
		% Right caption:
		\begin{subfigure}[t]{\widefigurewidth}
			\caption{Parse trees with qtree}
			\centering\tiny{\url{https://www.ling.upenn.edu/advice/latex/qtree/}}
			\label{fig:qtree-example}
		\end{subfigure}
		\medskip

		% Top rules:
		\colrules
		% Left content: code listing:
		\begin{subfigure}{\widefigurewidth}
			\includegraphics[width=\linewidth]{graphics/prftree.png}
		\end{subfigure}
		\hspace{\widefiguregap}
		% Right content: image or rendered example:
		\begin{subfigure}{\widefigurewidth}
			\includegraphics[width=\linewidth]{graphics/benzene-ring.pdf}
		\end{subfigure}
		% Bottom rules:
		\colrules
		% Left caption:
		\begin{subfigure}[t]{\widefigurewidth}
			\caption{Proof trees with prftree}
			\centering\tiny{\url{https://ftp.gwdg.de/pub/ctan/macros/latex/contrib/prftree/}}
			\label{fig:prftree-example}
		\end{subfigure}
		\hspace{\widefiguregap}
		% Right caption:
		\begin{subfigure}[t]{\widefigurewidth}
			\caption{Chemical structural formulas with chemfig}
			\centering\tiny{\url{http://latex-cookbook.net/cookbook/examples/benzene-ring/}}
			\label{fig:chemfig-example}
		\end{subfigure}
		\medskip
	}
	% General caption:
	\caption{Examples for some packages}
	\label{fig:package-examples}
\end{figure}


\begin{description}
	\item[Indices]
		can be created automatically with \texttt{makeidx}.\footnote{\url{https://www.ctan.org/pkg/makeidx}}
		By using \mintinline{tex}{\index{…}} one can mark entries for their index. With \mintinline{tex}{\printindex} they are assembled within index with references.
	\item[Vector graphics]
		(\cref{fig:tikz-example})
				can be \enquote{drawn} directly in the \LaTeX{} source code with \texttt{TikZ} (recursive acronym for \emph{TikZ ist kein Zeichenprogramm}, in English: \emph{TikZ is not a drawing program}).\footnote{\url{https://www.ctan.org/pkg/pgf}}
		Caution: This package is very powerful, but not necessarily beginner-friendly.
		Before creating vector graphics from sratch, we recommend you to experiment with some of the examples at \TeX{}ample\footnote{\url{https://texample.net/tikz/examples/}}. 
		For certain use cases, there are special packages that are easier to handle than \enquote{raw} TikZ:
	\item[Parse trees]
		that divide sentences into their grammatical components (\cref{fig:qtree-example}) can be created with \texttt{qtree}.\footnote{\url{https://ctan.org/pkg/qtree}}
	\item[Proof trees,]
		that are often used in logics (\cref{fig:prftree-example}), can be drawn with the package \texttt{prftree}.\footnote{\url{https://www.ctan.org/pkg/prftree}}
	\item[Chemical structural formulas]
		(\cref{fig:chemfig-example})
		can, amongst others, be created with  \texttt{chemfig}.\footnote{\url{https://www.ctan.org/pkg/chemfig}}
	\item[Colors]
		for your documents are provided by \texttt{xcolor}.\footnote{\url{https://www.ctan.org/pkg/xcolor}}
	\item[Notes,]
		that you have made in your source code and that you cannot overlook can be created with \texttt{todonotes}.\footnote{\url{https://www.ctan.org/pkg/todonotes}}
		With the package, one can mark what they still \todo{Please do not change. This is an example.} have to change within their document.
	\item[Pages of other \acro{PDF} files]
		can be integrated into the source code with \texttt{pdfpages}.\footnote{\url{https://www.ctan.org/pkg/pdfpages}}
		It comes in very handy whenever one needs the output of external programs in the document, for example, in within the appendix.
		Just compile the document one more time and the appendix is up-to-date again, if the external program has changed something.
	\item[Nested graphics]
		and the positioning of captions at almost any place are provieded by  \texttt{subcaption}.\footnote{\url{https://www.ctan.org/pkg/subcaption}}
		We also made extensive use of this package.
	\item[Tables]
		can be designed much more flexible than what we have shown here. 
		The following packages can help you with that:
		\texttt{colortbl},\footnote{\url{https://www.ctan.org/pkg/colortbl}}
		\texttt{tabularx},\footnote{\url{https://www.ctan.org/pkg/tabularx}}
		\texttt{multirow},\footnote{\url{https://www.ctan.org/pkg/multirow}}
		\texttt{makecell}.\footnote{\url{https://www.ctan.org/pkg/makecell}}
\end{description}

\noindent \texttt{beamer}, which is not a package, but another document class, can be used to create \textbf{slide shows}
with \LaTeX{}. Information on the document class and examples are available at Overleaf\footnote{\url{https://www.overleaf.com/learn/latex/Beamer}}, which brings us to the next section:

\section{Help and information}

\textbf{Wikibooks} provides you with a much more detailed introduction into \LaTeX{}. Note that the German version\footnote{\url{https://de.wikibooks.org/wiki/LaTeX-Kompendium}} is less complete than the English one.\footnote{\url{https://en.wikibooks.org/wiki/LaTeX}}
If required, both refer to additional packages.

Whenever you need information on certain packages \acro{\textbf{CTAN}}\footnote{\url{https://ctan.org/}} is your place to go. The official documentation as \acro{PDF} for each package can be found there.
Within this file, the first paragraphs are the most interesting. They are followed by implementation details, that your normally do not need.

If the official documentation is too theoretical, and you prefer a more hands-on approach, \textbf{Overleaf}\footnote{\url{https://www.overleaf.com/}} can help you out.
Primarily, it is a collaborative online \LaTeX{} editor. However, you can find multiple templates\footnote{\url{https://www.overleaf.com/latex/templates}} for different types of documents (VCs, theses, \textellipsis) there.

If you are looking for examples dedicated to TikZ, \textbf{\TeX{}ample}\footnote{\url{https://texample.net/}} provides you with multiple of them.

For concrete questions, the question-answering platform \textbf{Stackexchange} is a good place to go: There even is a \TeX{} community there.\footnote{\url{https://tex.stackexchange.com/}}

Needless to say, you can always contact us with your questions:
\begin{compactitem}
	\item via mail to \href{mailto:fachschaft-wiai.stuve@uni-bamberg.de}{fachschaft-wiai.stuve@uni-bamberg.de},
	\item via phone at +49951\,863\,1219
	\item or just come to our bureau at WE5/02.104.
\end{compactitem}

