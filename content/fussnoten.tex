\section{Fußnoten}

Für Fußnoten gibt es den Befehl \mintinline{latex}{\footnote{text}}, der an der Stelle des Aufrufs automatisch die richtige Ziffer einfügt und den übergebenen Text in der Fußzeile erscheinen lässt.
In Kombination mit dem Paket \mintinline{latex}{hyperref} sind die Fußnoten sowie URLs\footnote{sofern sie über den Befehl \mintinline{latex}{\url{\textellipsis}} gesetzt wurden}zudem anklickbar.
\todo{An welcher Stelle sollte die Einführung in das hyperref-Paket bestenfalls erscheinen?}

Das Paket \mintinline{latex}{footmisc} stellt verschiedene weitere Optionen für die Darstellung von Fußnoten zur Verfügung, die als optionale Parameter an den Befehl \mintinline{latex}{\usepackage} übergeben werden können.
\mintinline{latex}{\usepackage[perpage]{footmisc}} sorgt dafür, dass die Zählung der Fußnoten auf jeder Seite neu beginnt.
\mintinline{latex}{\usepackage[para]{footmisc}} lässt die Fußnoten in der Fußzeile als Fließtext erscheinen und \mintinline{latex}{\usepackage[symbol]{footmisc}} bewirkt eine Nummerierung mit Symbolen statt Ziffern.

\todo{Beispiel}

