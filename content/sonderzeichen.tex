\section{Sonderzeichen}

\subsection{Leerzeichen}

Das am häufigsten benötigte Sonderzeichen ist ein einfaches Leerzeichen als Wortzwischenraum.
Dieses Zeichen entsteht in \LaTeX, wenn im Quellcode zwischen anderen Zeichen Leerzeichen oder einzelne Zeilenumbrüche stehen.
Es gibt aber auch einige andere Arten von Leerzeichen:

\todo[inline]{Beispiel aus den Folien, Code + Bild}

\paragraph{English Spacing}
In englischsprachigen Dokumenten verwendet LaTeX das traditionelle English spacing, also doppelte Leerzeichen nach dem Satzende.
Der Befehl \texttt{\textbackslash frenchspacing} vor dem ersten Absatz verhindert das.
\texttt{\textbackslash nonfrenchspacing} schaltet wieder zurück.
Bei English Spacing versucht \LaTeX, Abkürzungen zu erkennen und danach trotzdem normale Wortzwischenräume zu setzen.
Das solltet ihr aber kontrollieren – und wo nötig,
Wortzwischenräume (\texttt{.\textbackslash\textvisiblespace}) oder Satzenden (\texttt{\textbackslash@.\textvisiblespace}) erzwingen.

\subsection{Striche}
Es gibt verschiedene horizontale Striche.
Am häufigsten werden der Bindestrich (-), der Halbgeviertstrich (–) und der Geviertstrich\footnote{Ein Geviert ist die Zeilenhöhe, die eine Bleiletter einnimmt} (—) verwendet.
Diese drei Striche werden in \LaTeX{} durch unterschiedlich viele Bindestriche im Quelltext erzeugt:
\begin{center}
	\begin{longtable}{llp{9cm}}
		\toprule
		Code       & Strich            & Verwendung \\
		\midrule
		\verb|-|   & Bindestrich       & \LaTeX-Wochenende \\
		\verb|--|  & Halbgeviertstrich & als Gedankenstrich – mit Leerzeichen außenrum – oder als Bis-Strich: 10\,–12\,Uhr \\
		\verb|---| & Geviertstrich     & a dash\,—\,mostly used in American English \\
		\bottomrule
	\end{longtable}
\end{center}
Der Bindestrich wird zwar auch zur Silbentrennung verwendet, ihr solltet ihn dafür aber nicht explizit im Quellcode eingeben.
An den meisten Stellen trennt \LaTeX{} automatisch richtig, wenn ihr das passende \texttt{babel}-Paket eingebunden habt.
Wenn dabei etwas schiefgeht, könnt ihr mit folgenden Codes\footnote{Auch dafür muss \texttt{babel} eingebunden sein.} eingreifen:
\begin{center}
	\begin{longtable}{lp{13cm}}
		\toprule
		Code      & Erläuterung \\
		\midrule
		\endhead
		\verb|\-| & Ausschließliche Trennstelle: Das Wort darf nur an dieser Stelle getrennt werden (kann auch mehrfach in einem Wort vorkommen, dann sind alle Stellen erlaubt). \\
		\verb|"-| & Zusätzliche Trennstelle: Das Wort darf neben den von \LaTeX{} erkannten Trennstellen auch hier getrennt werden. \\
		\verb|-|  & Exklusiver Bindestrich: Wird für Komposita verwendet und unterbindet die automatische Silbentrennung im Wort (das ist typografisch erwünscht). \\
		\verb|"=| & Nichtexklusiver Bindestrich: Wird für lange Komposita verwendet, bei denen die Silbentrennung aktiv bleiben soll, weil der Umbruch sonst nicht klappt. \\
		\verb|""| & Trennstelle ohne Strich: Kann beispielsweise verwendet werden, um \textsc{url}s ohne Bindestrich zu trennen. \\
		\verb|"~| & Bindestrich ohne Trennstelle: Wird verwendet, um den Bindestrich zusammen mit dem folgenden Wort umbrechen zu lassen: \emph{Vorlesungszeit und "~raum} \\
		\bottomrule
	\end{longtable}
\end{center}

\subsection{Anführungszeichen}
Anführungszeichen können mit folgenden Codes erzeugt werden.
Entscheidend ist dabei jeweils das Aussehen und nicht die Semantik, weshalb zum Beispiel im Deutschen die französichen Guillemets verkehrt herum verwendet werden (»french left quoation mark« rechts und umgekehrt):
\begin{center}
	\begin{longtable}{lcccc}
		\toprule
		Sprache & \multicolumn{2}{l}{Erste Ebene} & \multicolumn{2}{l}{Zweite Ebene} \\
		\midrule
		Deutsch & \verb|\glqq...\grqq| & \glqq…\grqq & \verb|\glq…\grq| & \glq…\grq \\
		Deutsch alternativ & \verb|\frqq...\flqq| & \frqq…\flqq & \verb|\frq…\flq| & \frq…\flq \\
		Englisch (\acro{A.\,E.}) & \verb|``...''| & ``…'' & \verb|`...'| & `…' \\
		Englisch (\acro{B.\,E.}) & \verb|`...'| & `…' & \verb|``...''| & ``…'' \\
		\bottomrule
	\end{longtable}
\end{center}
Deutlich flexibler seid ihr mit dem Paket \texttt{csquotes}, das den Befehl \texttt{\textbackslash{}enquote\{<zitat>\}} zur Verfügung stellt.
Abhängig von der Sprache werden bei die passenden Anführungszeichen verwendet, bei verschachtelten \texttt{enquote}s wird zwischen erster und zweiter Ebene gewechselt.
Mit der Option \texttt{autostyle=true} beim Paketimport setzt der Befehl \texttt{\textbackslash{}foreignquote\{<sprache>\}\{<zitat\}} je nach Sprache abweichende Anführungszeichen.

\subsection{Diakritika}
Wenn ihr Buchstaben mit diakritischen Zeichen direkt über die Tastatur eingeben könnt – beispielsweise die deutschen Umlaute oder gängige Akzente auf einer deutschen Tastatur – könnt ihr das auch direkt im Quelltext tun.
Die Zeichen bleiben dann im Output erhalten.
Ansonsten können die Diakritika auch per Escape-Codes erzeugt werden.
Die folgende Tabelle gibt nur einige Beispiele an – die Buchstaben lassen sich natürlich austauschen, es gibt aber auch noch viele andere Diakritika.
\begin{center}
	\begin{longtable}{lccclccclcc}
		\toprule
			\verb|\`{o}| & -- & \`{o} & $\quad$ & \verb|\c{c}| & -- & \c{c} & $\quad$ & \verb|\d{u}| & -- & \d{u} \\
			\verb|\'{o}| & -- & \'{o} & & \verb|\k{a}| & -- & \k{a} & & \verb|\r{a}| & -- & \r{a} \\
			\verb|\^{o}| & -- & \^{o} & & \verb|\l{}| & -- & \l{} & & \verb|\u{o}| & -- & \u{o} \\
			\verb|\"{o}| & -- & \"{o} & & \verb|\={o}| & -- & \={o} & & \verb|\v{s}| & -- & \v{s} \\
			\verb|\H{o}| & -- & \H{o} & & \verb|\b{o}| & -- & \b{o} & & \verb|\t{oo}| & -- & \t{oo} \\
			\verb|\~{o}| & -- & \~{o} & & \verb|\.{o}| & -- & \.{o} & & \verb|\o| & -- & \o \\
		\bottomrule
	\end{longtable}
\end{center}

\subsection{Andere Sonderzeichen}
Einige Sonderzeichen, beispielsweise das Prozentzeichen, sind für die \LaTeX{}-Syntax reserviert und können nicht als normale Zeichen verwendet werden.
Diese und viele, viele weitere Sonderzeichen können über eigene Befehle erzeugt werden.
Dabei ist zu beachten, dass manche Sonderzeichen nur in Mathe-Umgebungen\todo{Querverweis} funktionieren, andere brauchen zusätzliche Pakete.
\begin{center}
	\begin{longtable}{cll}
		\toprule
			?`/!` & \verb|?`/!`| & \\
			\textasciicircum & \verb|\textasciicircum| & \\
			\textasciitilde & \verb|\textasciitilde| & \\
			\textasteriskcentered & \verb|\textasteriskcentered| & \\
			\textbackslash & \verb|\textbackslash| & \\
			%\textbar & \verb|\textbar| & \\
			%\textbullet & \verb|\textbullet| & \\
			\textcopyright & \verb|\textcopyright| & \\
			\textdagger & \verb|\textdagger| & \\
			%\textdaggerdbl & \verb|\textdaggerdbl| & \\
			\textellipsis & \verb|\textellipsis| & \\
			\textless/\textgreater & \verb|\textless/\textgreater| & \\
			\textperthousand & \verb|\textperthousand| & \\
			\textsection & \verb|\textsection| & \\
			$\delta, \pi, \Sigma$ & \verb|\delta, \pi, \Sigma|, … & nur in Mathe-Umgebung \\
			\euro & \verb|\euro| & nur mit Paket \texttt{eurosym} \\
			\textteshlig & \verb|\textteshlig| & nur mit Paket \texttt{tipa} \\
			\textmusicalnote & \verb|\textmusicalnote| & nur mit Paket \texttt{textcomp} \\
		\bottomrule
	\end{longtable}
\end{center}
Falls ihr mal ein Sonderzeichen braucht, von dem ihr nicht genau wisst, wie es heißt, hilft euch \emph{Detexify}\footnote{\url{http://detexify.kirelabs.org/classify.html}} – ihr könnt das Symbol zeichnen und bekommt alle nötigen Infos.
Dass es von Keilschrift bis zu technischen Symbolen wirklich \emph{alles} gibt, stellt ihr fest, wenn ihr durch die \emph{Comprehensive \LaTeX Symbol List}\footnote{\url{http://tug.ctan.org/info/symbols/comprehensive/symbols-a4.pdf}} blättert.

