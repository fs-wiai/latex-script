\chapter{References}
\label{sec:references}

If we want to make references, like \enquote{\textellipsis , which you can see in figure 21, \textellipsis}, \LaTeX{} by default provides as with the command \mintinline{latex}{\ref{label}}. No more adapting of the numbering for graphics, tables, etc. needed!
The command expects a unique label as argument, that needs to be assigned to 
the referenced element. After that, wherever we call the command, the number of 
our referenced object appears in the text.
\Example{lst:fooboar-learning-latex}{references/ref}{references/ref_crop}{Example for a reference}

A smarter package for references is \texttt{cleverref}.\footnote{with only one \enquote{r}!}
It provides us with the command \mintinline{latex}{\cref{label}}, which can also handle multiple labels separated by commas.
This automatically generates text like \enquote{sections 1 to 3, and 5}.\footnote{for the source code \mintinline{latex}{\cref{sec:section1,sec:section2,
sec:section3,sec:section5}}}
Furthermore \mintinline{latex}{\cref{label}} automatically inserts a suited abbreviaton, e.\,g., \enquote{fig.} for figures.

We can reference graphics, tables, sections, chapters, source code listings, and equations. 
Many packages use the label in order to find out the object type of the referenced element.
For this reason, it is common to insert a prefix before each label (\cref{lst:reference-prefixes}).

\begin{table}[H]
  \centering
  \begin{tabular}{@{}llll@{}}
    \toprule
    Prefix & Object Type & Prefix & Object Type \\ \midrule
    fig: & figures & tbl:    & tables            \\ 
    sec: & sections  & subsec: & subsections     \\
    ch:  & chapters     & itm:    & items \\
    eq:  & equations & lst:    & source Code Listings  \\ \bottomrule
  \end{tabular}
  \caption{Prefixes for labels}
  \label{lst:reference-prefixes}
\end{table}

Note that, if we use \mintinline{latex}{\cref{label}}, for some document classes, the generated passages only appear in the desired language (e.\,g., German) when the language is specified already within the document class command:

\begin{minted}{latex}
\documentclass[ngerman]{article}
\end{minted}

	\noindent Except for sections, captions\footnote{\mintinline{latex}{\caption{…}}} \emph{always need to be specified and positioned before the label}. Otherwise, they cannot be referenced later on in the text.
	Labels for sections are inserted directly after the command:

\begin{minted}{latex}
\section{Comments}\label{sec:hints}
\end{minted}
