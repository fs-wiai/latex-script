\chapter*{Erste Schritte mit \LaTeX}
Bevor wir ins Arbeiten mit \LaTeX{} einsteigen können, müssen wir es natürlich erst mal installieren. Achtet bitte darauf, dass ihr \emph{zuerst} den Compiler und danach einen Editor installiert.

\section*{Compiler}
Zunächst zum Compiler. Ihn brauchen wir, um den Quelltext, den wir schreiben, in ein PDF umzuwandeln. Hier gibt es für verschiedene Betriebssysteme unterschiedliche Compiler, wie etwa MikTeX für Windows,\footnote{\url{https://miktex.org/download}} MacTex für MacOS\footnote{\url{http://tug.org/mactex/}} und TeXLive für Linux-Distributionen.\footnote{Sofern ihr eine debianbasierte Linux-Distribution verwendet, kann der Compiler mittles \mintinline{bash}{sudo apt install texlive-full} installiert werden. Für andere Distributionen findet ihr eine Anleitung unter \url{https://tug.org/texlive/doc/texlive-en/texlive-en.html\#installation}.} Bestenfalls installiert ihr die volle Version mit allen Paketen.

\section*{Editor}
Sobald ihr das gemacht habt, könnt ihr euch auch schon einen Editor herunterladen, in dem ihr eure \LaTeX-Dokumente gerne schreiben möchtet. Dafür ist eigentlich jeder Editor geeignet (notepad++, Atom, VS Code, usw.). Wir empfehlen für Anfänger jedoch ein Programm, das \LaTeX-spezfische Funktionen besitzt, wie etwa TeXstudio.\footnote{Eine aktuelle Version findet ihr unter \url{https://www.texstudio.org/}.}

\section*{Das erste Mal Kompilieren}
Öffnet nun die Datei \mintinline{bash}{main.tex} und kompiliert sie durch Druck auf \faForward. Was ist in dem Ordner passiert, in dem die Datei liegt?
