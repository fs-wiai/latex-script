\chapter{Mathematical Formulas}
\label{sec:maths}

Mathematical formulas are always set in \emph{math mode}, which, within a paragraph (referred to as \emph{inline}), can be turned on or off with a dollar sign.
There is also a \emph{block} environment (cf. \cref{lst:sample-math-environment}).
Important packages for mathematical features are the \texttt{amsmath}, 
\texttt{amsthm}, and \texttt{amssymb} packages of the American Mathematical 
Society, as well as \texttt{mathtools}.
As with many other environments, adding an asterisk turns off the numbering.

\example{lst:sample-math-environment}{maths/sample-math-environment}{Exemplary math environments}

\section{A few examples}
%Die \cref{tbl:maths-common-commands,tbl:maths-logic-sets-braces} listen einige häufig in Formeln verwendeten Befehle auf.
\todo{In diesem und dem nächsten Abschnitt fehlt mir ein bisschen die Orientierung durch Fließtext. Könnten wir nicht den Text der Tabellenunterschrift in einen Absatz vor der Tabelle umwandeln und die Tabellenunterschrift entsprechend kürzer halten?}

\begin{table}[H]
  \centering
  \begin{tabular}{ll}
  \toprule
  Source code & Result \\ \midrule
  \mintinline{latex}{\sqrt{16}} & $\sqrt{16}$ \\
  \mintinline{latex}{\frac{3}{4}} & $\frac{3}{4}$ \\
  \mintinline{latex}{e^{\pi}} & $e^{\pi}$ \\
  \mintinline{latex}{\sum_{i=1}\^{n}x^2} & $\sum_{i=1}^{n}x^2$ \\
  \mintinline{latex}{12 \leq 4 x^2 + 13} & $12 \leq 4 x^2 + 13$ \\
  \mintinline{latex}{{n \choose k}} & ${n \choose k}$ \\
  \bottomrule
  \end{tabular}
	\caption{Frequently used commands (square root, fraction, power, sum, inequation, binomial coefficient). 
	By \mintinline{latex}{^{…}} and \mintinline{latex}{_{…}}, the content is set in super- or subscript.}
  \label{tbl:maths-common-commands}
\end{table}

\begin{table}[H]
  \widebox{
    \centering
    \begin{tabular}{ll}
    \toprule
    Source code & Result \\ \midrule
    \mintinline{latex}{(x), [x], \lbrace x \rbrace, \lvert x \rvert} & $(x), [x], \lbrace x\rbrace, \lvert x\rvert$\\
    \mintinline{latex}{\exists,\forall,\in,\notin,\infty} & $\exists,\forall,\in,\notin,\infty$ \\
    \mintinline{latex}{\alpha, \beta, \Gamma, \Delta, \varepsilon, \pi} & $\alpha, \beta, \Gamma, \Delta, \varepsilon, \pi$ \\
    \mintinline{latex}{\rightarrow, \leftarrow, \Rightarrow, \Leftarrow, \Leftrightarrow} & $\rightarrow, \leftarrow, \Rightarrow, \Leftarrow, \Leftrightarrow$ \\
    \mintinline{latex}{(A \cup B) \cap C} & $(A \cup B) \cap C$ \\
    \mintinline{latex}{(A \lor B) \land C} & $(A \lor B) \land C$ \\
    \mintinline{latex}{(A \cdot B) \times C} & $(A \cdot B) \times C$ \\ \bottomrule
    \end{tabular}
    \caption{Brackets, quantifiers, greek letters, arrows, operators}
    \label{tbl:maths-logic-sets-braces}
  }
\end{table}

\section{Growing brackets}

Especially in combination with fractions, brackets should grow according to their content.
This can be achieved by pre-pendingnot each bracket (\mintinline{latex}{(}, \mintinline{latex}{)}, \mintinline{latex}{[}, \mintinline{latex}{]}, \mintinline{latex}{\lbrace} and \mintinline{latex}{\rbrace}) with a position marker (\mintinline{latex}{\left} oder \mintinline{latex}{\right}).

\example{lst:growing-brackets}{maths/growing-brackets}{Example for growing brackets}

\section{Lower and upper bounds}

The \mintinline{latex}{\limits} command renders lower and upper bounds of integrals above and below the integral sign.
Sums, products, and \todo{quite confusing in English}limits do this 
automatically (c.\,f. \cref{lst:limits}).
For inline formulas, \mintinline{latex}{\limits} are of limited suitability.

\example{lst:limits}{maths/limits}{Lower and upper bounds of sums, products, limits and integrals}

\section{Aligning equations}

The \texttt{align} environment allows to align multiple equations horizontally, e.\,g. at the \texttt{=} sign (\cref{lst:math-alignment-example}).
As in tables, the \texttt{\&} sign is used to specify anchorage points.
Line breaks are denoted by two backslashes.

\Example{lst:math-alignment-example}{maths/align-example}{maths/align-example_crop}{Equations aligned at equals signs}

\section{Intensional set notation}\todo{Is that really what this section is about? Looks rather like a “text within maths” section to me.}\todo{Ich kenne es nur als »Set-builder notation«, »text within maths« ist natürlich auch nicht falsch.}\todo{Diese Syntax mit den geschweifen Klammern und dem Strich, ja. Aber hier geht es doch eher um das mathrm, oder? Und das braucht man ja zum Beispiel auch für so ne Formeldefintion mit Fallunterscheidung und verbaler Beschreibung.}

Sometimes sets have to be defined in terms of textual descriptions or longer function names.
The \LaTeX{} math mode assumes that letters are variables rather than text, which creates problems when they are indeed supposed to be entire words.
For this case, there is the \mintinline{latex}{\mathrm{}} command (\enquote{math roman}, c.\,f. \cref{lst:set-builder-notation}).

\example{lst:set-builder-notation}{maths/set-builder-notation}{Problems arising from intensional set notation and their solution}
