\chapter{Mathematical Formulas}
\label{sec:maths}

Mathematical formulas are always set in \emph{math mode}, which, within a paragraph (referred to as \emph{inline}), can be turned on or off with a dollar sign.
There is also a \emph{block} environment (cf. \cref{lst:sample-math-environment}).
Important packages for mathematical features are the \texttt{amsmath}, 
\pkg{amsthm}, and \pkg{amssymb} packages of the American Mathematical 
Society, as well as \pkg{mathtools}.
As with many other environments, adding an asterisk turns off the numbering.

\example{lst:sample-math-environment}{maths/sample-math-environment}{Exemplary math environments}

\section{A few examples}
%Die \cref{tbl:maths-common-commands,tbl:maths-logic-sets-braces} listen einige häufig in Formeln verwendeten Befehle auf.

\begin{table}[H]
  \centering
  \begin{tabular}{ll}
  \toprule
  Source code & Result \\ \midrule
  \code{latex}{\textbackslash sqrt\{16\}} & $\sqrt{16}$ \\
  \code{latex}{\textbackslash frac\{3\}\{4\}} & $\frac{3}{4}$ \\
  \code{latex}{e\textasciicircum\{\textbackslash pi\}} & $e^{\pi}$ \\
  \code{latex}{\textbackslash sum\_\{i=1\}\textasciicircum\{n\}x\textasciicircum 2} & $\sum_{i=1}^{n}x^2$ \\
  \code{latex}{12 \textbackslash leq 4 x\textasciicircum 2 + 13} & $12 \leq 4 x^2 + 13$ \\
  \code{latex}{{n \textbackslash choose k}} & ${n \choose k}$ \\
  \bottomrule
  \end{tabular}
	\caption{Frequently used commands (square root, fraction, power, sum, inequation, binomial coefficient). 
	By \code{latex}{\textasciicircum\{…\}} and \code{latex}{\_\{…\}}, the content is set in super- or subscript.}
  \label{tbl:maths-common-commands}
\end{table}

\begin{table}[H]
  \widebox{
    \centering
    \begin{tabular}{ll}
    \toprule
    Source code & Result \\ \midrule
    \code{latex}{(x), [x], \textbackslash lbrace x \textbackslash rbrace, \textbackslash lvert x \textbackslash rvert} & $(x), [x], \lbrace x\rbrace, \lvert x\rvert$\\
    \code{latex}{\textbackslash exists,\textbackslash forall,\textbackslash in,\textbackslash notin,\textbackslash infty} & $\exists,\forall,\in,\notin,\infty$ \\
    \code{latex}{\textbackslash alpha, \textbackslash beta, \textbackslash Gamma, \textbackslash Delta, \textbackslash varepsilon, \textbackslash pi} & $\alpha, \beta, \Gamma, \Delta, \varepsilon, \pi$ \\
    \code{latex}{\textbackslash rightarrow, \textbackslash leftarrow, \textbackslash Rightarrow, \textbackslash Leftarrow, \textbackslash Leftrightarrow} & $\rightarrow, \leftarrow, \Rightarrow, \Leftarrow, \Leftrightarrow$ \\
    \code{latex}{(A \textbackslash cup B) \textbackslash cap C} & $(A \cup B) \cap C$ \\
    \code{latex}{(A \textbackslash lor B) \textbackslash land C} & $(A \lor B) \land C$ \\
    \code{latex}{(A \textbackslash cdot B) \textbackslash times C} & $(A \cdot B) \times C$ \\ \bottomrule
    \end{tabular}
    \caption{Brackets, quantifiers, greek letters, arrows, operators}
    \label{tbl:maths-logic-sets-braces}
  }
\end{table}

\section{Growing brackets}

Especially in combination with fractions, brackets should grow according to their content.
This can be achieved by pre-pending each bracket (\code{latex}{(}, \code{latex}{)}, \code{latex}{[}, \code{latex}{]}, \code{latex}{\textbackslash lbrace} and \code{latex}{\textbackslash rbrace}) with a position marker (\code{latex}{\textbackslash left} or \code{latex}{\textbackslash right}).

\example{lst:growing-brackets}{maths/growing-brackets}{Example for growing brackets}

\section{Lower and upper bounds}

The \code{latex}{\textbackslash limits} command renders lower and upper bounds of integrals above and below the integral sign.
Sums, products, and limits do this 
automatically (cf. \cref{lst:limits}).
For inline formulas, \code{latex}{\textbackslash limits} are of bounded suitability.

\example{lst:limits}{maths/limits}{Lower and upper bounds of sums, products, limits and integrals}

\section{Aligning equations}

The \mono{align} environment allows to align multiple equations horizontally, e.\,g., at the \mono{=} sign (\cref{lst:math-alignment-example}).
As in tables, the \mono{\&} sign is used to specify anchorage points.
Line breaks are denoted by two backslashes.

\Example{lst:math-alignment-example}{maths/align-example}{maths/align-example_crop}{Equations aligned at equals signs}

\section{Text in math mode}

Sometimes sets have to be defined in terms of textual descriptions or longer function names.
The \LaTeX{} math mode assumes that letters are variables rather than text, which creates problems when they are indeed supposed to be entire words.
For this case, there is the \code{latex}{\textbackslash mathrm\{\}} command (\enquote{math roman}, c.\,f. \cref{lst:set-builder-notation}).

\example{lst:set-builder-notation}{maths/set-builder-notation}{Problems arising from intensional set notation and their solution}
