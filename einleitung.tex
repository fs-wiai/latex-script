% Siehe https://github.com/texdoc/l2kurz

\begin{titlepage}
\renewcommand{\thefootnote}{\fnsymbol{footnote}}
{\Huge%
\fontfamily{lmss}\fontseries{sbc}\selectfont
\raggedright
\sbLaTeXe-Kurzbeschreibung
\rule{\textwidth}{0.75pt}
\par
}
\begin{flushleft}
  \normalsize
  \fontfamily{lmss}\fontseries{sbc}\selectfont
  Version \lkver\\
  \lkdate\\[2ex]
  Marco Daniel\\
  Patrick Gundlach\\
  Walter Schmidt\\
  Jörg Knappen\\
  Hubert Partl%\footnote{Zentraler Informatikdienst der Universität für Bodenkultur, Wien}
    \\
  Irene Hyna%\footnote{Bundesministerium für Wissenschaft und Verkehr, Wien}
  \\
\end{flushleft}

\vfill

{\parindent=0cm
\LaTeX{} ist ein Satzsystem, das für viele Arten von
Schriftstücken verwendet werden kann, von einfachen Briefen bis zu
kompletten Büchern.  Besonders geeignet ist es für
wissenschaftliche oder technische Dokumente. \LaTeX{} ist für
praktisch alle verbreiteten Betriebssysteme verfügbar.
\bigskip

Die vorliegende Kurzbeschreibung bezieht sich auf die Version
\LaTeXe\ in der Fassung vom Juni~2001 und sollte für den
Einstieg in \LaTeX{} ausreichen.
Eine vollständige Beschreibung enthält das \manual{}
in Verbindung mit der Online-Dokumentation.
}
\setcounter{footnote}{0}
\end{titlepage}


{\parindent=0cm\thispagestyle{empty}

Autoren: 1998--2016 M.~Daniel, P.~Gundlach, W.~Schmidt, J.~Knappen, H.~Partl, I.~Hyna

\bigskip

% Lizenzänderung in Absprache mit Walter Schmidt <-> Patrick Gundlach 19. Juni 2012
{\selectlanguage{USenglish}
This material may be distributed only subject to the terms and
conditions set forth in the \emph{Open Publication License}, v1.0 or
later (the latest version is presently available at
\url{http://www.opencontent.org/openpub/}).}


\bigskip

Die in dieser Publikation erwähnten Software- und Hardware"=Bezeichnungen sind
in den meisten Fällen auch eingetragene Warenzeichen und unterliegen als
solche den gesetzlichen Bestimmungen.

\bigskip

\vfill

Dieses Dokument wurde mit \LaTeX{} gesetzt.
Es ist als Quelltext und im PDF-Format online erhältlich:
\begin{quote}
\url{http://mirror.ctan.org/info/lshort/german/}
\end{quote}
Die Änderungen seit Version 2.3 (10.\,April 2003) sind unter \url{https://github.com/texdoc/l2kurz} einzusehen.
\bigskip

Die Autoren bedanken sich bei
Luzia Dietsche,
Michael Hofmann,
Peter Karp,
Rolf \mbox{Niepraschk},
Heiko Oberdiek,
Bernd Raichle,
Rainer Schöpf und
Stefan Steffens
für Tipps, Anmerkungen und  Korrekturen.

}
