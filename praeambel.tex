\documentclass[a4paper, 12pt, ngerman]{scrartcl}

\usepackage[utf8]{inputenc}
\usepackage[T1]{fontenc}
\usepackage{babel}

\usepackage{eurosym, tipa, textcomp, textgreek, upgreek} % Sonderzeichen

\usepackage{amssymb, amsfonts, amsmath} % Mathezeug

\usepackage{hyperref} % Referenzen als Links
\hypersetup{colorlinks,breaklinks,urlcolor=blue,linkcolor=blue}
\urlstyle{same} % URLs werden in derzeitiger Schrift statt dicktengleich ausgegeben

\usepackage{cleveref} % mehr mit Referenzen

\usepackage{graphicx} 
\graphicspath{graphics/}
\usepackage{subcaption}

\usepackage{paralist} % kompakte Listen

\usepackage{longtable, array, tabularx, booktabs, colortbl} % Tabellenzeug

\usepackage{todonotes}

\usepackage{minted} % Quelltext-Listings

\usepackage{csquotes} % für \enquote

\newcommand\acro[1]{\textsc{\lowercase{#1}}}

% arguments:
% 1. label
% 2. content path
% 3. caption
\newcommand\example[3]{
	\begin{figure}[H]
		\begin{subfigure}[b]{.48\textwidth}
			\hrule\medskip
			\inputminted[breaklines]{tex}{content/#2}
			\hrule\medskip
			\caption{\LaTeX-Code}
			\label{#1-code}
		\end{subfigure}
		\hfill
		\begin{subfigure}[b]{.48\textwidth}
			\hrule\medskip
			\input{content/#2}
			\hrule\medskip
			\caption{Ergebnis}
			\label{#1-result}
		\end{subfigure}
		\caption{#3}
		\label{#1}
	\end{figure}

}

