\newpage
\definecolor{latexblue}{rgb}{0.9,0.925,0.95}
\pagecolor{latexblue}

\todo{Refactor text for new exercise script.}

\chapter*{First steps with \LaTeX}
\addcontentsline{toc}{section}{First steps with \LaTeX}

This script serves as a short \LaTeX{} reference and as exercise material for the Fachschaft \acro{WIAI} \LaTeX{} workshop.
All of the source files and the latest version of this script can be found on Github.\footnote{\url{https://github.com/fs-wiai/latex-script/releases}}

The following instructions will equip you with the programs necessary to create documents with \LaTeX{}---they will be explained in the following chapters.
Please, make sure to install the \emph{compiler first} and the \emph{editor afterwards}\textit{.}

\section*{Compiler}
Let’s start with the compiler.
(Seriously!)
We will need it to convert our \LaTeX{} documents into \acro{PDF}s.
There are different compilers for different operating systems;
for example, MikTeX for Windows,\footnote{\url{https://miktex.org/download}} Mac\TeX{} for macOS,\footnote{\url{http://tug.org/mactex/}} and \TeX{}Live for Linux distributions.\footnote{On Debian-based Linux distributions, install the compiler by executing \sh{sudo apt install texlive-full.}
For other distributions, see \url{https://tug.org/texlive/doc/texlive-en/texlive-en.html\#installation}.}
In case you get to choose, it is best to install the full version with all packages.

\section*{Editor}
As soon as you have installed the compiler, you can download an editor that you are going to use to write your \LaTeX{} documents.
Any editor will do (notepad++, Atom, VS Code, etc.).
However, for beginners, we recommend using \TeX{}studio,\footnote{You find the latest version on \url{https://www.texstudio.org/}.} a program that supports you with \LaTeX-specific features.

\section*{Compiling for the first time}
Open up the file \file{main-exercises.tex} in \TeX{}studio. 
It can be found in our project directory.
By pressing \includegraphics[width=.8em]{graphics/faForward.png}, it is converted into a a file called \file{main.pdf}.
You should be able to find the latter by looking at the project directory in your file explorer.
If your installation is not working yet, \TeX{}studio will show you an error message.
Feel free to contact us in this case.\footnote{Find us at \url{https://www.uni-bamberg.de/wiai/fs}.}
Otherwise, you are now ready to go!

\newpage
\nopagecolor
