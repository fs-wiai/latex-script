\begin{enumerate}
	\item 
	 \textbf{Put the sections of the file into separate files, named 
	 \file{section1.tex}, \file{section2.tex}, and \file{section3.tex}. Include 
	 them using the  \code{latex}{\textbackslash include} command.}
		
	\begin{figure}[H]
		\codeblock{latex}{exercises/project-structure/main-with-preamble.done.tex}
		\caption{\file{main.tex}}
	\end{figure}
	\begin{figure}[H]
		\codeblock{latex}{exercises/project-structure/section1.done.tex}
		 \caption{\file{section1.tex} (analogous for the other 
			sections)}
	\end{figure}
 	\item \textbf{Which command becomes superfluous when you use 
	\code{latex}{\textbackslash include}?} \\
	\code{latex}{\textbackslash include} makes every included file appear on a 
	new page. Hence the command \code{latex}{\textbackslash newpage} becomes 
	superfluous.
	
	\item \textbf{Can the preamble also be excluded? If no, why? If yes, when 
	can outsourcing the preamble be useful?} 

	 The preamble can also be outsourced, like in task 1. In real-world 
	 \LaTeX{} projects we often make use of numerous packages that need to be 
	 configured by additional commands within the preamble. Therefore, it is 
	 advisable to outsource the preamble. On top of that, this is useful 
	 because the file with the most-commonly used packages can be moved from 
	 one project to another without needing to rewrite the preamble every time.
	 
	\item 
	\textbf{Add the command \code{latex}{\textbackslash 
	includeonly\{section2\}} to the preamble. Compile the document again, and 
	check what has changed. What does the command do and how can it be helpful 
	in a larger project? }

	The command \code{latex}{\textbackslash includeonly\{section2\}} does 
	exactly what the name already says. During compilation, only the 
	\code{latex}{\textbackslash include} statements that contain file names 
	that appear in \code{latex}{\textbackslash includeonly\{file1,file2, …\}} 
	are included.
	
	This can be useful for larger projects, when the compile time becomes 
	longer since by using the \code{latex}{\textbackslash includeonly} only the 
	files can be included which we are currently working in. Before the final 
	compilation we can delete the \code{latex}{\textbackslash includeonly} 
	command and end up with our complete \acro{PDF}.
\end{enumerate}