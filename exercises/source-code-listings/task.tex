Im Ordner \mintinline{text}{exercises/source-code-listings} findet ihr eine Datei namens \mintinline{text}{Source.java}.
Wir werden diese im Folgenden in unser Dokument einbinden und das Aussehen unseren Wünschen anpassen.

\begin{enumerate}
  \item Bindet die Datei an dieser Stelle ein.
  \item Aktiviert Syntax-Highlighting durch Angabe der Sprache Java.
  \item Lasst LaTeX zu lange Zeilen automatisch umbrechen.
  \item Die Zeilen sollen nummeriert sein.
  \item Nutzt das Theme \mintinline{latex}|native|.
  \item Dieses Theme ist für einen dunklen Hintergrund optimiert. Ändert die Hintergrundfarbe des Themes zu dunkelblau.
  \item Bindet jetzt ausschließlich die Zeilen 5 bis 7 ein.
  \item Entfernt die Leerzeichen am Anfang der Zeilen durch Angabe der richtigen Option. (Hinweis: Die Dokumentation spricht hier von gobble.)
\end{enumerate}

Konsultiert bei Fragen die Dokumentation des Paketes Minted.

\exercisematerial{exercises/source-code-listings/task-1}