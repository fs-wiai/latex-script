You will find a file called \mintinline{text}{Source.java} in the folder \mintinline{text}{exercises/source-code-listings}.
We will now include it into our document and adjust its display to fit our needs.

\begin{enumerate}
  \item Include the file into \mintinline{text}{exercises/source-code-listings/source-code-listings.tex}.
  \item Activate syntax highlighting by stating the programming language Java.
  \item Add line breaks and line numbers.
  \item Use the theme \mintinline{latex}|native|.
  \item This theme is optimized for a dark background. Change the background color to dark blue.
  \item Include only lines 5 to 7.
  \item Delete the spaces at the beginning of the lines by using a suitable option. (Hint: The documentation speaks of \mintinline{latex}|gobble|.)
\end{enumerate}

\noindent If you have questions, consult the minted package documentation.

\exercisematerial{exercises/source-code-listings/source-code-listings}
\usemintedstyle{} % Reset experiments from exercise material
