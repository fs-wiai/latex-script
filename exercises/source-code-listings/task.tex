In the folder \file{exercises/source-code-listings}, you will find a file called \file{Source.java}.
We will now include it into our document and adjust its display to fit our needs.
If you have questions, consult the \pkg{minted} or \pkg{listings} package documentation.

\section*{Minted tasks}

\begin{enumerate}
  \item Include the file into \\
  \file{exercises/source-code-listings/source-code-listings.tex}.
  \item Activate syntax highlighting by stating the programming language Java.
  \item Add line breaks and line numbers.
  \item Use the theme \code{latex}{native}.
  \item This theme is optimized for a dark background. Change the background color to dark blue.
  \item Include only lines 5 to 7.
  \item Delete the spaces at the beginning of the lines by using a suitable 
  option. (Hint: The documentation speaks of \code{latex}{gobble}.)
\end{enumerate}

\section*{Listings tasks}

\begin{enumerate}
  \item Include the file into \\
  \file{exercises/source-code-listings/source-code-listings.tex}.
  \item Activate syntax highlighting by stating the programming language Java.
  \item Set the \mono{basicstyle} to a proper mono-spaced font (\mono{\textbackslash ttfamily \textbackslash small})
  \item Add line numbers.
  \item Change the keyword color to blue.
  \item Don't show special characters for spaces in strings.
\end{enumerate}

\exercisematerial{exercises/source-code-listings/source-code-listings}

% Reset experiments from exercise material if minted is used
\ifthenelse{\equal{\listingsmode}{minted}}{%
  \usemintedstyle{default} 
}{}
