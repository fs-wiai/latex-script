\begin{enumerate}
	\item Im Ordner \mintinline{latex}{exercises/basic-document-structure} findet ihr eine Datei namens \mintinline{latex}{document-structure.tex}. Öffnet die Datei, kopiert den Text und fügt ihn in eine neue Datei ein, die ihr beispielsweise \mintinline[breaklines, breakbytokenanywhere]{latex}{document- structure-solution.tex} nennt. Packt den Text in der neuen Datei in eine \mintinline{latex}{document}-Umgebung und fügt eine Präambel ein, um euer erstes \LaTeX -Dokument anschließend kompilieren zu können. Kompiliert nun das neue Dokument. 
	\item Herzlichen Glückwunsch, ihr habt euer erstes eigenes \LaTeX -Dokument erstellt und kompiliert. Wie euch vielleicht aufgefallen ist, sind die Absätze im Dokument mit \mintinline{latex}{\\} erstellt worden. Ersetzt diese durch richtige Absätze.
	\item Nun ist es an der Zeit, das Dokument etwas zu strukturieren. Verwendet für die Überschriften die passenden \LaTeX -Befehle (\mintinline{latex}{\section}, \mintinline{latex}{\subsection}, usw.) und fügt anschließend ein Inhaltsverzeichnis in euer Dokument ein.
	\item Kommentiert im Anschluss die Präambel, die Dokumentenumgebung und das Inhaltsverzeichnis wieder aus. Nur so kann eure Lösung auch ins Skript eingebunden werden.
\end{enumerate}
\exercisematerial{exercises/basic-document-structure/document-structure}
