% \documentclass{scrartcl}
% \usepackage[utf8]{inputenc}
% \usepackage[T1]{fontenc}
% \usepackage[english]{babel}
% \usepackage[hidelinks]{hyperref}
	
% \begin{document}
% \tableofcontents
	
\section{Beverages}
	
\subsection{Pink Lemonade}
We all know and love it in summer: a cold lemonade. The pink variant of our  favorite summer drink is traditionally achieved by adding food coloring to the lemonade. 
However, we changed things up a little and decided to go for a variant with berries instead of food coloring, which also makes the drink less sour.
	
\subsubsection{Ingredients}
For making the lemonade we need 400\,g of berries. You can use either frozen or fresh ones. You can, of course, adjust the choice of berries to your likings: raspberries, blueberries, blackberries, or a mixture work excellently for making the lemonade.

To make the drink sweeter we use 50\,ml of maple syrup. Needless to say, you can also use less, if you want your lemonade to be more sour. By the way, other sweeteners, such as agave syrup, or coconut sugar also work fine.

Additionally, we need the juice of four freshly-squeezed lemons, one litre worth of cold water and 400\,g of ice cubes to keep the drink cold.
	
\subsubsection{Instructions}
Put the berries along with the maple syrup into a large bowl and cook them at medium-high heat for around three to four minutes. When the berries start to soften, reduce the heat a little. Now mash the berries with the end of a wooden spoon until there are almost no big chunks anymore. The less chunks the better!
	
Place a sieve over the container that you want to store the lemonade in. Put the berries into the sieve. With the end of the wooden spoon, try to press out as much liquid of the berry mixture as possible. Let the juice sit until it is completely cold.
	
When the liquid is cold, add the lemon juice, the cold water, and the ice cubes and stir everything. 
	
	
Your lemonade is now ready to enjoy!
	
\subsection{Hot Chocolate}
Coming soon!
	
\section{Breakfast}

\subsubsection{Buckwheat Overnight Oats}
Coming soon!
	
% \end{document}
