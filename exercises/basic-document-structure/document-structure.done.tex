\begin{minted}[breaklines]{latex}
\documentclass{scrartcl}
\usepackage[utf8]{inputenc}
\usepackage[T1]{fontenc}
\usepackage[ngerman]{babel}
\usepackage{hyperref}
\usepackage{csquotes}
	
\begin{document}
\tableofcontents
	
\section{Getränke}
	
\subsection{Pink Lemonade}
Wir alle kennen und lieben sie im Sommer: eine kühle Limonade. Die pinke Variante wurde traditionell zubereitet, indem man der Limonade Lebensmittelfarbe hinzugefügt hat. Wir haben das Ganze etwas abgewandelt und uns für eine Variante mit Beeren entschieden, um den sauren Geschmack der Zitronen auszugleichen. Dadurch wird die Limonade ein perfekter und erfrischender Sommer-Drink!
	
\subsubsection{Zutaten}
Für die Zubereitung der Limonade braucht ihr 400 g Beeren. Sie können entweder gefroren oder frisch sein. Bei der Wahl der Beeren könnt ihr euch austoben: Himbeeren, Heidelbeeren, Brombeeren oder auch eine Mischung aus verschiedenen Beeren eignen sich hervorragend. 
	
Für ein wenig Süße nutzen wir 50 ml Ahornsirup, natürlich könnt ihr auch mehr oder weniger nehmen, je nachdem, wie sauer ihr eure Limonade haben wollt. Übrigens eignen sich auch andere Süßungsmittel wie etwa Agavendicksaft oder Kokosblütensirup wunderbar.
	
Außerdem benötigen wir noch 4 gepresste Zitronen, ungefähr einen Liter kaltes Wasser und 400 g Eiswürfel, damit das Getränk auch schön kühl bleibt.
	
\subsubsection{Zubereitung}
Die Beeren zusammen mit dem Ahornsirup in einen Kochtopf geben und bei mittlerer Hitze für drei bis vier Minuten kochen. Sobald die Beeren anfangen, weich zu werden, könnt ihr die Hitze etwas reduzieren. Mit dem Ende eines Holzlöffels zerdrückt ihr nun die Beeren, bis so gut wie keine großen Stücke mehr vorhanden sind. Je weniger Stücke, desto besser!
	
Nun platziert ihr ein Sieb über dem Gefäß, in dem ihr eure Limonade aufbewahren wollt, und gebt die Himbeeren in das Sieb. Mit der Rückseite eines Löffels versucht ihr, so viel Saft wie möglich aus der Beeren-Mixtur zu pressen. Den Saft danach abkühlen lassen.
	
Sobald eure Beeren-Flüssigkeit abgekühlt ist, gebt den Zitronensaft, das kalte Wasser und die Eiswürfel hinzu und rührt alles ordentlich um.
	
Eure Limonade ist nun fertig zum genießen und bereit, in Gläser gefüllt zu werden!
	
\subsection{Heiße Schokolade}
Coming soon!
	
\section{Frühstücksgerichte}
\subsubsection{Overnight Oats aus Couscous}
Coming soon!
	
\end{document}
\end{minted}