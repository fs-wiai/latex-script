
	1. Getr"anke
	
	1.1 Pink Lemonade
	Wir alle kennen und lieben sie im Sommer: eine k"uhle Limonade. Die pinke Variante wurde traditionell zubereitet, indem man der Limonade Lebensmittelfarbe hinzugef"ugt hat. Wir haben das ganze etwas abgewandelt und uns f"ur eine Variante mit Beeren entschieden, um den sauren Geschmack der Zitronen auszugleichen. Dadurch wird die Limonade ein perfekter und erfrischender Sommer-Drink!
	
	1.1.1 Zutaten
	F"ur die Zubereitung der Limonade braucht ihr 400 g Beeren. Sie k"onnen entweder gefroren oder frisch sein. Bei der Wahl der Beeren k"onnt ihr euch austoben: Himbeeren, Heidelbeeren, Brombeeren oder auch eine Mischung aus verschiedenen Beeren eignen sich hervorragend. \\
	F"ur ein wenig S"u"se nutzen wir 50 ml Ahornsirup, nat"urlich k"onnt ihr auch mehr oder weniger nehmen, je nachdem, wie sauer ihr eure Limonade haben wollt. "Ubrigens eignen sich auch andere S"u"sungsmittel wie etwa Agavendicksaft oder Kokosbl"utensirup wunderbar. \\
	Au"serdem ben"otigen wir noch 4 gepresste Zitronen, ungef"ahr einen Liter kaltes Wasser und 400 g Eisw"urfel, damit das Getr"ank auch sch"on k"uhl bleibt..
	
	1.1.2 Zubereitung
	Die Beeren zusammen mit dem Ahornsirup in einen Kochtopf geben und bei mittlerer Hitze f"ur drei bis vier Minuten kochen. Sobald die Beeren anfangen, weich zu werden, k"onnt ihr die Hitze etwas reduzieren. Mit dem Ende eins Holzl"offels zerdr"uckt ihr nun die Beeren, bis so gut wie keine gro"sen St"ucke mehr vorhanden sind. Je weniger St"ucke, desto besser! \\
	Nun platziert ihr ein Sieb "uber das Gef"a"s, in dem ihr eure Limonade aufbewahren wollt und gebt die Himbeeren in das Sieb. Mit der R"uckseite eines L"offels versucht ihr, so viel Saft wie m"oglich aus der Beeren-Mixtur zu pressen. Den Saft danach abk"uhlen lassen. \\
	Sobald eure Beeren-Fl"ussigkeit abgek"uhlt ist, gebt den Zitronensaft, das kalte Wasser und die Eisw"urfel hinzu und r"uhrt alles ordentlich um. \\
	Eure Limonade ist nun fertig zum genie"sen und bereit, in Gl"aser gef"ullt zu werden!
	
	1.2 Hei"se Schokolade
	... coming soon!
	
	2. Fr"uhstucksgerichte
	
	2.1 Overnight Oats aus Couscous
	... coming soon!
	