Die \emph{Rekursion} ist eine bekannte Problemlösestrategie in der Informatik. Dabei wird ein großes Problem zerlegt und zunächst nur ein Teil der Aufgabe gelöst. Oft wird \emph{Rekursion} innerhalb von Programmcode umgesetzt, indem eine Methode sich immer wieder selbst aufruft, bis eine bestimmte Abbruchbedingung erfült ist. Das heißt, dass das Programm theoretisch bis ins Unendliche weiterlaufen kann, wenn es \emph{rekursiv} ist.

Weite Informationen findet ihr im Wikipedia-Artikel zur Rekursion unter \url{https://de.wikipedia.org/wiki/Rekursion}.